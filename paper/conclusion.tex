\section{Conclusions and Perspectives}

The development CPNs has been driven by a research agenda with
simultaneous focus on theoretical development, design and
implementation of software tool support, and practical application and
case studies. These three aspects has had a clear mutual influence in
that as a theoretical foundation is needed in order to develop
semantically sound software tools which in turn is needed in order to
get evidence on the practical applications and limitations of the
theoretical foundation. In addition, an international standard for
high-level Petri nets was developed headed by Billington
\cite{hcpnstandard} and approved in 2004. The high-level Petri net
standard is heavily based on CPNs which adhere to the standard.

\ignore{
An example of this is also that it was during
the creation of the computer tool support that the need of modules was
discovered and it was also in this phase that it became clear that the
type concept from programming languages was extremely adequate for
type definitions and net inscriptions (instead of using more ad-hoc
notations).}

The development of the CPN modeling languages can be divided into
four steps which has many similarities with the developments in
programming languages. The first step from the black tokens of basic
Petri nets to colored tokens in PrT-nets corresponds to the step from
bit representation to simple data types. The second step from PrT-nets
to CPNs corresponds to the invention of structured data types and type
checking. The third step with the introduction of hierarchical CPNs
corresponds to the introduction of structuring concepts like modules,
procedures, functions and subroutines. The fourth step represented by
the development of Design/CPN and CPN Tools corresponds to the
implementation of compilers and run-time systems which are essential
for programming languages in a similar way as a Petri net language is
of no practical use without tool support. 

In this paper we have provided an overview and an introduction to the
CPN language and its computer tool support and we have discussed the
research development that led to the development of the CPN modeling
language as it exists today. The most recent monograph on Colored
Petri Nets has been published by Jensen and Kristensen in 2009
\cite{newcpnbook}. It provides an in-depth, but yet compact
introduction to modeling and validation of concurrent systems by means
of CPNs. The book introduces the constructs of the CPN modeling
language, presents its analysis methods, and provides a comprehensive
road map to the practical use of CPNs. Furthermore, the book presents
some selected industrial case studies illustrating the practical use
of CPN modeling and validation for design, specification, simulation,
and verification in a variety of application domains. The book is
aimed at use both in university courses and for self-study.

\ignore{
THEORY - TOOLS - PRACTICE?

Can be learned with reading the formal definitions. Functional prgramming has been an obstrace - but this is changing with the increased poplurao of funcational programming langauges such as Scala ...

Do we want to mention the expressiveness (theoretical) of CPNs?

State sapce of CPN models are often smaller because we can do complex data manipulation without introdcuing internmediate steps.

% future directions

The primary analysis of CPN models has been based on dynanmic analysis
by means of simulation or state space exploration. Static analysis of
CPN models involves combining stuctural analysis of the net structure
(i.e., places, transitions, and arcs) and static analysis of the
inscriptions (arc expressions and guards). Hence, static analysis of
CPN models must be combined with status analysis as known from
programming langauges. This is a research direction that has been
pursued only to a very limited extent.

EXTENSION TO THE MODELLING LANGUAGE? AMPLE OPPERTUNITES FOR
INTEGRARINF MORE RECENT CONCEPT FROM PROG: LANGUAGE INTO CPNS.

CODE GENERATION
}
