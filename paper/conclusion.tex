\section{Conclusions and Perspectives}

Research into CPNs has been driven by an agenda with simultaneous
focus on theoretical development, design and implementation of
software tool support, and practical application and case
studies. These three aspects have had a clear mutual influence on each
other. A theoretical foundation is needed in order to develop
semantically sound software tools which in turn is needed for
practical applications which reveal limitations of the theoretical
foundation.\rdc{ The development of mature tool support in the form of
Design/CPN and CPN Tools has been absolutely essentialy to the
widespread use of CPNs.}


\ignore{
An example of this is also that it was during
the creation of the computer tool support that the need of modules was
discovered and it was also in this phase that it became clear that the
type concept from programming languages was extremely adequate for
type definitions and net inscriptions (instead of using more ad-hoc
notations).}

\rdc{
The creation of the CPN language can be divided into four steps which
have many similarities with the developments in programming
languages. The first step from the black tokens of basic Petri nets to
colored tokens in PrT nets corresponds to the step from bit
representation to simple data types. The second step from PrT nets to
CPNs corresponds to the invention of structured data types and type
checking. The third step with the introduction of hierarchical CPNs
corresponds to the introduction of structuring concepts like modules,
procedures, functions and subroutines. The fourth step represented by
the development of Design/CPN and CPN Tools corresponds to the
implementation of compilers and run-time systems which are essential
for practical applications. }

In addition to CPNs, several other variants of high-level Petri nets
and supporting computer tools have been developed based on the idea of
extending Petri Nets with data types. One example is Well-Formed Nets
(WFNs) \cite{wfns} as supported, e.g., by the CosyVerif tool. WFNs put
rather strong restrictions on the colour sets and associated
operations to facilitate space-efficient exploration of state spaces
in which the nodes correspond to equivalence classes of states instead
of single states. Concurrent Object-Oriented Nets \cite{coopn} as
implemented in the COOPNBuilder tool incorporate object-oriented
concepts into Petri nets and relies on Algebraic Petri Nets (APNs)
\cite{apn} for specification of data types and inscriptions. APNs are
high-level Petri Nets in which algebraic abstract data types are used
for giving semantics to the inscriptions. Reference Nets (RNs)
\cite{rns} are high-level Petri nets targeting agent-oriented object
systems in which the tokens themselves may constitute references to
other Petri net models. RNs are supported by the Renew tool which uses
Java as the inscription language. Stochastic Well-formed Nets
\cite{swns} as supported, e.g., by the GreatSPN tool supports
performance evaluation based on Markov chains. An international
standard for high-level Petri nets was developed headed by Billington
\cite{hcpnstandard} and approved in 2004. The high-level Petri net
standard is heavily based on CPNs which conforms to the standard.



In this paper we have provided an overview and an introduction to the
CPN language and its tool support, and we have discussed the research
development that led to the development of the CPN modeling language
as it exists today. The most recent monograph on Colored Petri Nets
has been published by Jensen and Kristensen in 2009
\cite{newcpnbook}. It provides an in-depth, but yet compact
introduction to modeling and validation of concurrent systems by means
of CPNs. The book introduces the constructs of the CPN modeling
language, presents its analysis methods, and provides a comprehensive
road map to the practical use of CPNs. Furthermore, the book presents
selected industrial case studies illustrating the practical use of
CPNs for design, specification, simulation, and verification in a
variety of application domains. The book is aimed at use both in
university courses \cite{teaching} and for self-study.

\ignore{
THEORY - TOOLS - PRACTICE?

Can be learned with reading the formal definitions. Functional prgramming has been an obstrace - but this is changing with the increased poplurao of funcational programming langauges such as Scala ...

Do we want to mention the expressiveness (theoretical) of CPNs?

State sapce of CPN models are often smaller because we can do complex data manipulation without introdcuing internmediate steps.

% future directions

The primary analysis of CPN models has been based on dynanmic analysis
by means of simulation or state space exploration. Static analysis of
CPN models involves combining stuctural analysis of the net structure
(i.e., places, transitions, and arcs) and static analysis of the
inscriptions (arc expressions and guards). Hence, static analysis of
CPN models must be combined with status analysis as known from
programming langauges. This is a research direction that has been
pursued only to a very limited extent.

EXTENSION TO THE MODELLING LANGUAGE? AMPLE OPPERTUNITES FOR
INTEGRARINF MORE RECENT CONCEPT FROM PROG: LANGUAGE INTO CPNS.

CODE GENERATION
}
