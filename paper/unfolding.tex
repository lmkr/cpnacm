\section{Unfolding and Folding of CPNs}

% unfolding and folding
The extensions of Petri nets that CPNs bring in the form of data
types, a programming language, and modules add practical modeling
power to Petri nets by making it possible to create models of complex
real systems. The extensions do not add expressive power from a
theoretical perspective as any hierarchical CPN model can be unfolded
to a non-hierarchical CPN model which in turn can be \concept{unfolded}
to a behaviorally equivalent (and possibly infinite) Place/Transition
net (PTN). The unfolding of a hierarchical CPN to a non-hierarchical CPN
consists of recursively replacing each substitution transition with
its associated submodule such that related port and socket places are
merged into a single place. The unfolding of a non-hierarchical CPN to
a PTN consists of unfolding each CPN place to a PTN place for each
color in the color set of the CPN place, and unfolding each CPN
transition to a PTN transition for each possible binding of the CPN
transition. In the other direction, any PTN can be \concept{folded}
into a CPN with a single module consisting of a single place and a
single transition. In practice such a folding is not interesting,
because the arc expressions will be extremely complex and
non-interpretable for a human being. However, the fact that the
unfolding and folding exists shows that hierarchical CPNs has the same
theoretical properties as basic Petri nets -- in particular that CPNs
constitute a solid model for concurrency, conflict, synchronization
and resource sharing.

%Anything which can be programming in Java can (in theory) also be
%programmed in assembler code. It is just much more time-consuming and
%much more error-prone. 

%Analogously it can be proved (quite easily) that each hierarchical CPN
%can be unfolded to a (much larger) basic Petri net (Place Transition
%Net) with exactly the same dynamic behavior.

% illustrate place and transition unfolding

To illustrate the unfolding consider the fragment shown in
Fig.~\ref{fig:sendcancommit}. To represent this fragment as an
ordinary Petri net, the CPN place \figitem{CanCommit} needs to be
unfolded to PTN places corresponding to \smlcode{wrk(1)} and
\smlcode{wrk(2)}. The place \figitem{Idle} and \figitem{WaitingVotes}
does not need to be unfolded as the \smlcode{Unit} color set contains
just a single value. Similarly, no unfolding of the
\figitem{SendCanCommit} transition is required in this case since the
transition does not have any variables. The equivalent PTN is shown in
Fig,~\ref{fig:sendcancommitunfold}. It can be seen that the arc
expressions are replaced by arc weights, and that the initial marking
is replaced by the specification of a single token initially in
\figitem{Idle}. To show the unfolding of a CPN transition consider the
CPN model fragment shown in Fig.~\ref{fig:receivecancommit}. To
represent this as a PTN, we need to unfold the places as explained in
the previous paragraph and in addition unfold the transition
\figitem{ReceiveCanCommit} to a PTN transition for each of the four
binding elements listed in
Fig.~\ref{fig:bindingelements}. Figure~\ref{fig:receivecancommitunfold}
shows the corresponding PTN for worker one. The PTN will have an
identical copy also for worker two.


% illustrate transition unfolding

\begin{figure}[b]
\centering
\includegraphics[scale=.45]{figures/PTSendCanCommit.eps}
\caption{PTN representation of the CPN in Fig.~\ref{fig:sendcancommit}.}
\label{fig:sendcancommitunfold}
\end{figure}


\begin{figure}[t]
\centering
\includegraphics[scale=.45]{figures/PTReceiveCanCommit.eps}
\caption{PTN representation corresponding to worker one for the CPN in Fig.~\ref{fig:receivecancommit}.}
\label{fig:receivecancommitunfold}
\end{figure}

% It can be seen that we essentially
%need a copy of the CPN fragment for each worker in the system.

The unfolding above also demonstrates that Petri nets do not provide a
way to easily scale the model according to some system parameter (in
this case the number of workers). With ordinary Petri nets is is
necessary to have a subnet for each worker (even though they behave in
exactly the same way). With PrT nets and CPNs we can use tokens with
color \smlcode{wrk(1)} to model the state of the first worker, tokens
with the color \smlcode{wrk(2)} to model the state of the second
worker, and so on. This means that we for \smlcode{W} workers can have
a single \figitem{Idle} place, which may contain tokens of \smlcode{W}
different colors -- instead of having a separate \smlcode{Idle} place
for each of the \smlcode{W} workers. In particular, in the CPN model
where we just need to change the symbolic constant \smlcode{W} (see
Fig.~\ref{fig:coloursets}) to configure the model to handle, e.g.,
five workers whereas with basic Petri nets we need to add places,
transitions, and arcs and hence change the net structure in order to
increase the number of workers.  This shows that CPNs provides a means
for easily creating parameterizable models and also that it enables
more compact modeling as we only need a single instance of the
\figitem{CanCommit} place in order to accommodate any finite number of
workers. Comparing CPN and PrT nets. With CPN it is possible to have
many color sets and hence we can use a number of different color sets
(e.g. one color set for the coordinator, a second for the workers, a
third for Yes/no votes and a fourth for abort/commit decisions). With
PrT nets only one set of token colors are allowed (or Cartesian
product thereof) and hence with PrT nets we could have had to
represent the identity workers, Yes/No votes and abort/commit
decisions by colors based on this single set of token colors.
