% THIS IS SIGPROC-SP.TEX - VERSION 3.1
% WORKS WITH V3.2SP OF ACM_PROC_ARTICLE-SP.CLS
% APRIL 2009
%
% It is an example file showing how to use the 'acm_proc_article-sp.cls' V3.2SP
% LaTeX2e document class file for Conference Proceedings submissions.
% ----------------------------------------------------------------------------------------------------------------
% This .tex file (and associated .cls V3.2SP) *DOES NOT* produce:
%       1) The Permission Statement
%       2) The Conference (location) Info information
%       3) The Copyright Line with ACM data
%       4) Page numbering
% ---------------------------------------------------------------------------------------------------------------
% It is an example which *does* use the .bib file (from which the .bbl file
% is produced).
% REMEMBER HOWEVER: After having produced the .bbl file,
% and prior to final submission,
% you need to 'insert'  your .bbl file into your source .tex file so as to provide
% ONE 'self-contained' source file.
%
% Questions regarding SIGS should be sent to
% Adrienne Griscti ---> griscti@acm.org
%
% Questions/suggestions regarding the guidelines, .tex and .cls files, etc. to
% Gerald Murray ---> murray@hq.acm.org
%
% For tracking purposes - this is V3.1SP - APRIL 2009

\documentclass{acm_proc_article-sp}

\usepackage{graphics}
\begin{document}

%\title{The Coloured Petri Nets Modeling Languages for Concurrent Systems}

\title{Coloured Petri Nets: A Graphical Language for Formal Modeling and Validation of Concurrent Systems}
\subtitle{}

%\titlenote{A full version of this paper is available as
%\textit{Author's Guide to Preparing ACM SIG Proceedings Using
%\LaTeX$2_\epsilon$\ and BibTeX} at
%\texttt{www.acm.org/eaddress.htm}}}
%
% You need the command \numberofauthors to handle the 'placement
% and alignment' of the authors beneath the title.
%
% For aesthetic reasons, we recommend 'three authors at a time'
% i.e. three 'name/affiliation blocks' be placed beneath the title.
%
% NOTE: You are NOT restricted in how many 'rows' of
% "name/affiliations" may appear. We just ask that you restrict
% the number of 'columns' to three.
%
% Because of the available 'opening page real-estate'
% we ask you to refrain from putting more than six authors
% (two rows with three columns) beneath the article title.
% More than six makes the first-page appear very cluttered indeed.
%
% Use the \alignauthor commands to handle the names
% and affiliations for an 'aesthetic maximum' of six authors.
% Add names, affiliations, addresses for
% the seventh etc. author(s) as the argument for the
% \additionalauthors command.
% These 'additional authors' will be output/set for you
% without further effort on your part as the last section in
% the body of your article BEFORE References or any Appendices.

\numberofauthors{2} %  in this sample file, there are a *total*
% of EIGHT authors. SIX appear on the 'first-page' (for formatting
% reasons) and the remaining two appear in the \additionalauthors section.
%
\author{
% You can go ahead and credit any number of authors here,
% e.g. one 'row of three' or two rows (consisting of one row of three
% and a second row of one, two or three).
%
% The command \alignauthor (no curly braces needed) should
% precede each author name, affiliation/snail-mail address and
% e-mail address. Additionally, tag each line of
% affiliation/address with \affaddr, and tag the
% e-mail address with \email.
%
% 1st. author
\alignauthor
Kurt Jensen\\
       \affaddr{Computer Science Department}\\
       \affaddr{Aarhus University, Denmark}\\
       \email{kjensen@cs.au.dk}
% 2nd. author
\alignauthor
Lars M. Kristensen\\
       \affaddr{Department of Computing}\\
       \affaddr{Bergen University College, Norway}\\
       \email{lmkr@hib.no}
}

\maketitle
\begin{abstract}

Coloured Petri Nets (CPNs) combine Petri nets with a programming
language to obtain a scalable formal modeling language for concurrent
systems. Petri nets provide the formal foundation for modeling
concurrency and synchronization, and a programming language provides
the primitives for modeling data manipulation and creating compact and
parameterizable models. We provide an example driven introduction to
the core syntactical and semantical constructs of the CPN modeling
language, and briefly surveys how quantitative and qualitative
behavioral properties of CPN models can be validated using
simulation-based performance analysis and explicit state space
exploration. In addition, we give a brief overview of CPN Tools which
provide tool support for the practical use of CPNs, and provide
pointers to some significant examples where the CPN technology has
been put into practical use in an industrial setting. As we proceed,
we provide a historical perspective on the research that led to the
development of the CPN language.

\end{abstract}

% A category with the (minimum) three required fields
\category{H.4}{Information Systems Applications}{Miscellaneous}
%A category including the fourth, optional field follows...
\category{D.2.8}{Software Engineering}{Metrics}[complexity measures, performance measures]

\terms{Theory}

\keywords{TODO} % NOT required for Proceedings

\newcommand{\com}[1]{
        \mbox{}
       \marginpar{\hrule\footnotesize\raggedright\hspace{0pt}#1\vspace{2mm}}
}

\newcommand{\todo}[1]{
       TODO: \mbox{}
       \marginpar{\hrule\footnotesize\raggedright\hspace{0pt}#1\vspace{2mm}}
}

\newcommand\concept[1]{{\em #1}}
\newcommand\figitem[1]{{\sf #1}}
\newcommand\smlcode[1]{{\tt{#1}}}
\newcommand\ignore[1]{}
\newcommand\fix[1]{\textbf{#1}?}

% for reduction in revised version
\newcommand\rdc[1]{}


\section{Introduction}

The vast majority of IT systems today can be characterized as
concurrent and distributed systems in that their operation inherently
relies on communication, synchronization, and resource sharing between
concurrently executing software components and applications. This is a
development that has been accelerated first with the pervasive
presence of the Internet as a communication infrastructure, and in
recent years by, e.g, cloud- and web-based services, mobile
applications, and multi-core computing architectures.

% main motivation and application domain of CPNs

The development of Coloured Petri Nets (CPNs) was initiated in the
early 80'es when distributed system was becoming a major paradigm for
future computing systems. The goal of the CPN modeling language was to
develop a formally founded modeling language for concurrent systems
that would make it possible to formally analyze and validate
concurrent systems, and which from a modeling perspective would scale
to industrial systems. A main motivation behind the research into CPNs
(and many other formal modeling languages) was that the engineering of
correct concurrent systems is a challenging task due to their complex
behavior which may result in subtle bugs if not carefully designed. As
concurrent systems are becoming still more pervasive and critical to
society, formal techniques for concurrent system was -- and still is
-- a highly relevant technology to support the engineering of reliable
concurrent systems.

% historical perspectic on development and roots

At its origin, CPNs builds on Petri nets (see sidebar on Petri nets)
that were introduced by Carl Adam Petri in his doctoral thesis
published in 1962 \cite{capetri:thesis} as a formalism for concurrency
and synchronization. The introduction of Petri Nets by C.A. Petri was
far ahead of the time where distributed systems were invented and
computers started to have parallel processes. At that time, programs
and processing were considered to be sequential and
deterministic. Hence, it was extremely visionary of C.A. Petri to
predict the importance of being able to understand and characterize
the basic concepts of concurrency. In Petri nets, concurrency is a
fundamental concept in that Petri nets is inherently based on the idea
that behavior is (implicitly) concurrent unless explicitly
synchronized. This is in contrast to many other modeling formalisms
where concurrency must be explicitly introduced using parallel
composition operators. A further advantage of Petri nets is that they
rely on very few basic concepts, and is still able to model a wide
range of communication and synchronization concepts and patterns.\\

%\noindent\rule{\columnwidth}{0.2em}
\paragraph*{\textsc{\textbf{SIDEBAR I: Petri Nets}}}
Petri nets in their basic form are called Place/Transition nets (PTNs)
and is a directed bi-partite graph with nodes consisting of places
(drawn as ellipses) and transitions (drawn as rectangles). The state
of a Petri is called a marking and consists of a distribution of
tokens (drawn as black dots) positioned on the places. The execution
of a Petri net (also referred to as the token game) consists of
occurrences of enabled transitions removed tokens from input places
and adding tokens to output places as described by integer arc weights
thereby changing the current state (marking) of the Petri
net.\hfill$\qed$ \\
%\noindent\rule{\columnwidth}{0.2em}
% disadvantages - need for further development

In the decade following the introduction by C.~A. Petri, Petri nets
were widely accepted as one of the most well-founded theories to
describe important behavioral concepts such as concurrency, conflict,
synchronization and resource sharing. Petri nets were also used to
model and analyze smaller concurrent systems. However, the practical
use soon revealed a serious shortcoming. Petri nets (in their basic
form) do not scale to large systems unless one models the systems at a
very high level of abstraction. The primary reasons for this is that
Petri nets are not well-suited for modeling systems in which data and
manipulation of data plays a crucial role. Furthermore, Petri nets did
not provide concepts that made it easy to scale models according to
some system parameter, e.g,. increase the number of servers in a
modeled system without having to make major changes to the model. This
implied that the use of Petri nets for practical modelling were
staggering. To remedy this situation many researchers proposed
different ad-hoc extensions to Petri nets. This created a large zoo of
different Petri net modeling languages. Many ad-hoc extensions were
not well-defined, and even when they were, they often had fundamental
theoretical problems. Whenever a new ad-hoc extension was introduced,
all the basic concepts and analysis methods had to be redefined -- to
apply for the extended Petri net language (with the ad-hoc
extension). With the invention of the first (text-based) computer
tools to support the analysis of Petri net models, the situation
became acute. Whenever a new ad-hoc extension was introduced (to
handle a modeling shortcoming) all existing computer tools became
void, and could only be used after time-consuming and error-prone
reprogramming. Hence, there was an urgent need to develop a class of
Petri nets that were general enough to handle a large variety of
different application areas without the need of making ad-hoc
extensions.

The first successful step towards a common more powerful class of
Petri nets were taken by Genrich and Lautenbach in 1979 with the
introduction of Predicate/Transition Nets (PrT nets)
\cite{genrich:81}. Their work was inspired by earlier work on
transition nets with \concept{colored tokens} by Schiffers and Wedde
\cite{schiffers:78} and transition nets with \concept{complex
  conditions} \cite{Y} by Shapiro. The basic idea behind PrT nets was
to introduce a set of colored tokens which can be distinguished from
each other -- in contrast to the indistinguishable black tokens in
basic Petri nets. In this way it became possible to model different
processes in a single subnet. PrT nets used arc expressions to define
how transitions can occur in different ways (occurrence modes)
depending on the colors of the involved input and output tokens. The
invention of colored distinguishable tokens in PrT nets was a gigantic
step forward -- but it still had some limitations. PrT nets only had
one set of token colors, and all places have to use this set (or
Cartesian products based on this set).

The second step towards a more general class of Petri nets was taken
by Jensen in his PhD thesis in 1980 \cite{jensen:81} with the
introduction of the first kind of Colored Petri Nets (CPNs). This
Petri net model allowed the modeler to use a number of different color
sets. This made it possible to represent data values in a more
intuitive way instead of having to encode all data into a single
shared set. It later turned out to be convenient to define the color
sets by means of data types known from programming languages, such as
products, records, lists, and enumerations. The use of types had three
implications: Token colors became structured (and hence much more
powerful); type checking became possible (making it much easier to
locate modeling errors); and color sets, arc expressions and guards
could be specified by the well-known and powerful syntax and semantics
known from programming languages. This gave the modeler a convenient
way to handle complex data and specify the often complex interaction
between data and system behavior. A third step forward was taken by
Huber, Jensen, and Shapiro in 1990 with the introduction of
Hierarchical CPNs \cite{huber:91}. Their work was heavily inspired by
the hierarchy concepts in the Structured Analysis and Design Technique
(SADT) developed by Marca and McGowan \cite{sadt}. It was Shapiro who got
the idea to port the SADT hierarchy concepts to CPN.  The introduction
of hierarchical CPNs allowed the modeler to structure a large CPN
model into a number of interacting and re-usable modules -- in a
similar way as known from programming languages. This implied that
Petri net models of large systems become much more tractable, since
they can be split into modules of a reasonable size and the model can
be viewed at different levels of abstraction.


\ignore{
The shortcoming of ordinary Petri nets outlined above prompted a
research direction into the development of high-level Petri nets which
... KURT TO ADD HISTORICAL PERSPECTIVE ON THE DEVELOPMENT OF
HIGH-LEVEL NETS. WE NEED TO TALK ABOUT STANDARD ML SOMEWHERE AND SAY
THAT CPN ML IS BASED ON SML.
}


\section{Coloured Petri Nets}

To present the concepts of CPNs, we use a CPN model of a distributed
two-phase commit transaction system. We use the example to give an
informal introduction to CPNs. The reader interested in the formal
definition of CPNs is referred to \cite{X}.

% modules - substitution transitions

The CPN model of the two-phase commit system is comprised of 4
\concept{modules} hierarchically organized into three
level. Figure~\ref{fig:commit} shows the top-level module which
consists of two \concept{substitution transitions} (drawn as
rectangles with double-lined borders) representing the
\figitem{Coordinator} and the \concept{Workers} in the system. Each of
the substitution transition has an associated \concept{submodule} that
model the behavior of the coordinator and the workers,
respectively. The name of an associated submodule is written in the
rectangular tag positioned at the bottom of each substitution
transition.

\begin{figure}[b]
\centering
\includegraphics[scale=.52]{figures/Commit.eps}
\caption{The top-level module of the CPN model.}
\label{fig:commit}
\end{figure}

% places - tokens - colours - and colour sets

The two substitution transitions are connected via directed
\concept{arcs} to the four places \figitem{CanCommit},
\figitem{Votes}, \figitem{Decision}, and \figitem{Acknowledge} (drawn
as ellipses). Places connected to substitution transitions are called
\concept{socket places} and are linked to \concept{port} places on the
associated submodules (to be presented shortly). The coordinator and
the workers interact by producing and consuming \concept{tokens} on
the places. These tokens carry data values and the type of tokens that
may reside on a place is determined by the \concept{color set} (type)
of the place (written in text below the
place). Figure~\ref{fig:coloursets} lists the definitions of the
color sets used as color set of the four places in
Fig.~\ref{fig:commit}. These color sets are defined using the CPN ML
programming language.

The \smlcode{Worker} color set is an indexed color set consisting of
the values \smlcode{wrk(1),wrk(2),...,wrk(W)} where the symbolic
constant \smlcode{W} is used to specify the number of worker processes
considered. This color set is used to model the identity of the
worker processes. The color set \smlcode{Vote} is an enumeration
color set containing the values \smlcode{Yes} and \smlcode{No}, and
is used to model that a worker by vote yes or no to commit the
transaction. The color set \smlcode{WorkerxVote} is a product type
containing pairs consisting of a worker and a vote. The color set
\smlcode{Decision} is an enumeration color set used to model whether
the coordinator decides to \smlcode{abort} or \smlcode{commit} the
transaction (only if all workers vote yes will the transaction be
committed). It should be noted that in addition to the type
constructors introduced above, CPN ML supports union, lists, and
record types. The variables \smlcode{w} and \smlcode{vote} declared in
Fig.~\ref{fig:coloursets} will be introduced later.

\begin{figure}[]
\begin{verbatim}
val W = 2;
colset Worker = index wrk with  1..W;
var w : Worker;

colset Vote        = with Yes | No;
var vote : Vote;
colset WorkerxVote = product Worker * Vote;

colset Decision        = with abort | commit;
colset WorkerxDecision = product Worker * Decision;
\end{verbatim}
\caption{Color sets definition used in Fig.~\ref{fig:commit}.}
\label{fig:coloursets}
\end{figure}


% marking - tokens initial marking - current marking

The state of a CPN model is called a \concept{marking}), and consists
of a distribution of \concept{tokens} on the places of the model. Each
place may hold a (possibly empty) \concept{multi-set} of tokens with
data values (colors) from the color set of the place. The
\concept{initial marking} of a place is specified above each place
(and omitted if the initial marking is the empty multi-set). For the
places in Fig.~\ref{fig:commit} all places are empty in the initial
marking. Initially, the \concept{current marking} of a CPN model
equals the initial marking. When a CPN model is executed occurrences of
enabled transitions consume and produce tokens on the places which
will change the \concept{current marking} of the CPN model.

 
% coordinator  - port places

The \figitem{Coordinator} module is shown in
Fig.~\ref{fig:coordinator}. This is the submodule associated with the
\figitem{Coordinator} submodule in Fig.~\ref{fig:commit}. The places
\figitem{CanCommit}, \figitem{Votes}, \figitem{Decision}, and
\figitem{Acknowledge} are \concept{port places} as indicated by the
rectangular \figitem{In} and \figitem{Out} tags positioned next to
them. These places are linked to the accordingly named places in the
top-level module (Fig.~\ref{fig:commit}) via a \concept{port-socket
  association} which implies that any tokens added/removed from a port
place by transitions in the \concept{Coordinator} module will also be
reflected in the marking of the associated socket place in the
top-level module. The places \figitem{CanCommit} and
\figitem{Decision} are \concept{output port places} which means that
the \figitem{Coordinator} module will only produce tokens on these
places. The places \figitem{Votes} and \figitem{Acknowledge} are
\concept{input port places} which means that the module will only
consume tokens from these places. It is also possible for a place to
be an input-output port place which means that the module may both
produce and consume tokens on (from) this place.

The places \figitem{Idle}, \figitem{WaitingVotes}, and
\figitem{WaitingAcknowledgement} are used to model the states that the
coordinator goes through when executing the two-phase commit
protocol. The places \figitem{Idle} and \figitem{WaitingVotes} have
the color set \smlcode{UNIT} containing just a single value
\smlcode{()} (denoted unit). Initially, the coordinator is in an idle
state as modeled by the initial marking of place
\figitem{Coordinator} which consist of a single token with the color
unit. In CPN ML this multi-set is written \smlcode{1`()} specifying
one (\smlcode{1}) occurrence of (\smlcode{`}) the unit color
(\smlcode{()}).  The number of tokens on a place in the current
marking is indicated with a small circle positioned next to a place,
and the detail of the color of the tokens are provided in an
associated text box. The indication of the current marking of a place
is omitted if currently the place contains no tokens.

The transitions \figitem{SendCanCommit}, \figitem{CollectVotes}, and
\figitem{ReceiveAcknowledge} model the event/actions that cause the
coordinator to change state. The coordinator will first send a can
commit message (transition \figitem{SendCanCommit}) to each worker
asking whether they can commit the transaction; then the coordinator
will collect the votes from all workers (transition
\figitem{CollectVotes}) and then send a decision to each worker
indicating whether the transaction is to be committed or not. Finally,
the coordinator will receive an acknowledgment from each worker that
they have received the decision (transition
\figitem{ReceiveAcknowledgements}).  It should be noted that
\figitem{CollectVotes} is a substitution transition which means that
the details of how the coordinator collects votes is modeled by the
associated \figitem{CollectVotes} submodule. This illustrates that it
is possible to mix the use of ordinary and substitution transitions
within a module.

\begin{figure}[]
\centering
\includegraphics[scale=.5]{figures/Coordinator.eps}
\caption{The \figitem{Coordinator} module.}
\label{fig:coordinator}
\end{figure}


% basic enabling and occurrences

In the current marking shown in Fig.~\ref{fig:coordinator} only the
transition \figitem{SendCanCommit} is enabled as indicated by the
thick border of that transition. The requirement for a transition to
be \concept{enabled} is determined from the \concept{arc expressions}
associated with the incoming arcs of the transition. In this case,
there is only a single incoming arc from place \figitem{Idle}
containing the expression \smlcode{()}. This expression specifies that
for \figitem{SendCanCommit} to be enabled, there must be at least one
\smlcode{()}-token present on \smlcode{Idle}. When the
\smlcode{SendCanCommit} transition \concept{occurs}, it will consume a
\smlcode{()}-token from place \smlcode{Idle} and it will produce
tokens on place connected to output arcs as determining by evaluating
the arc expressions on output arcs. In this case, the expression
\smlcode{()} on the arc to \smlcode{WaitingVotes} evaluates to a
single \smlcode{()}-token. The expression \smlcode{Worker.all ()} is a
function call when calls a function \smlcode{Worker.all} that takes a
unit value as argument and returns all colors of the color set
\smlcode{Worker}. This illustrates that arc expression may also make
use of function to calculate the tokens to be added (removed) from
places. In particular this means that complex data manipulation can be
performed without intruding intermediate steps in the model itself.

Figure~\ref{fig:sendcancommit} shows the marking of the surrounding
places of transition \figitem{SendCanCommit} after the occurrence of
\figitem{SendCanCommit}. It can be seen that the place
\smlcode{CanCommit} contains two tokens - one of each worker in the
system - representing messages going to the two worker processes. The
coordinator has now entered a state in which it is waiting to collect
the votes from the worker processes.

\begin{figure}[h]
\centering
\includegraphics[scale=.5]{figures/SendCanCommit.eps}
\caption{Current marking after \figitem{SendCanCommit}.}
\label{fig:sendcancommit}
\end{figure}

% comparison to low-level nets

By setting the symbolic constant \smlcode{W} (see
Fig.~\ref{fig:coloursets}) we can easily configure to model to handle,
e.g., five workers. With ordinary Petri nets we would have had to
create a copy of the \figitem{CanCommit} place for each worker. In
particular, we would have to make changes to the net structure
(places, transitions, arcs) when changing the number of workers. This
shows that CPNs provides a means for easily creating parameterizable
models and also that it enables more compact modeling as we only need
a single instance of the \figitem{CanCommit} place in order to
accommodate any finite number of workers. 

\begin{itemize}
\item WE COULD EABORATE MORE ON THIS BY DRAWING THE CORRESPONDING
  PT-NET FOR THE CANCOMMIT TRANSITION. THAT WOULD ILLUSTRATE UNFOLDING
  OF PLACES (see comment at the end of this section that proposes
  illustration of unfolding transition. Perhaps it is a good approach
  to split the introduction of unfolding in two steps. Then at the end
  of the section we can mention that any CPNs can be unfolder to a
  possibly infinite behaviorally equivalent PT-nets. The two example
  that we have in this section would then nicely illustrate how. It
  would also be possible to factor this into a seperate section in
  order to avoid that this section becomes overlong in comparison with
  the other section.  
\end{itemize}

%\com{Maybe we could have shown
%  the corresponding fragment as a Place/Transition net? In this case we need more places - when we get to the worker we could also need to duplicate transition as per bindings}

% worker process - and binding and binding elements.

Figure~\ref{fig:worker} shows the \figitem{Worker} module which is the
submodule of the \figitem{Workers} substitution transition in
Fig.~\ref{fig:commit}. The places \figitem{CanCommit},
\figitem{Votes}, \figitem{Decision}, and \figitem{Acknowledge}
constitute the port places of this module and are linked to the
accordingly named socket places in Fig.~\ref{fig:commit}. The places
\figitem{Idle} and \figitem{Waiting} models the two main states of
worker processes. Each of these places have the color set
\smlcode{Worker} and the idea is that when there is a token with
color \smlcode{wrk(i)} on, e.g., the place \figitem{Idle}, then this
represent that the i'th worker is in state idle. This makes it
possible to model the state of all workers in a compact manner within
a single module without having to have a place for each worker or a
module instance for each worker. Initially, all workers are in the
idle state as represented by corresponding tokens on place
\figitem{Idle} in the initial marking. The transition
\figitem{ReceiveCanCommit} models the reception of can commit messages
from the coordinator and the sending of a vote. The transition
\figitem{ReceiveDecision} models the reception of a decision message
from the coordinator and the sending of an acknowledgment.

\begin{figure}[]
\centering
\includegraphics[scale=.5]{figures/Worker.eps}
\caption{The \figitem{Worker} module.}
\label{fig:worker}
\end{figure}

The current marking of place \figitem{CanCommit} is \smlcode{1`wrk(1)
  ++ 1`wrk(2)} corresponding a marking where the coordinator has sent
a can commit message to each worker. The thick border of transition
\figitem{ReceiveCanCommit} indicates that this transition is enabled
in the current marking. The arc expressions on the surrounding arcs
of the \figitem{ReceiveCanCommit} transition are more complex than the
arc expressions of the \figitem{SendCanCommit} transition in the
\figitem{Coordinator} module considered earlier in that they contain
the \smlcode{free variables} \smlcode{w} and \smlcode{vote} defined in
Fig.~\ref{fig:coloursets}. This means that in order to talk about the
enabling and occurrence of transition \smlcode{ReceiveCanCommit}, we
need to assign values to these variables in order to evaluate the
input and output arc expressions. This is done by creating a
\concept{binding} which associates a value to each of the free
variables occurring in the arc expression of the transition. Bindings
can be considered different modes in which a transition may occur.  As
\smlcode{w} is of type \smlcode{Worker} and \smlcode{vote} is of type
\smlcode{Decision}, this gives the following four possible bindings
reflecting that each of the two workers may vote yes or no to
committing the transactions:

\begin{eqnarray*}
b_{1Y} & = & \langle \smlcode{w} = \smlcode{wrk(1)}, \smlcode{vote} = \smlcode{Yes} \rangle \\
b_{1N} & = & \langle \smlcode{w} = \smlcode{wrk(1)}, \smlcode{vote} = \smlcode{No} \rangle \\
b_{2Y} & = & \langle \smlcode{w} = \smlcode{wrk(2)}, \smlcode{vote} = \smlcode{Yes} \rangle \\
b_{2N} & = & \langle \smlcode{w} = \smlcode{wrk(2)}, \smlcode{vote} = \smlcode{No} \rangle
\end{eqnarray*}

A binding of a transition is enabled if evaluating each input arc
expression in the binding results in a multi-set of tokens which is a
subset of the multi-set of tokens present on the corresponding input
place. For an example, consider the binding $b_{1Y}$. Evaluating the
input arc expression \smlcode{w} on the input arc from \smlcode{Idle}
results in the multi-set containing a single token with the color
\smlcode{wrk(1)} which is contained in the multi-set of tokens present
on place \figitem{Idle} in the marking depicted in
Fig.~\ref{fig:worker}. Similarly for the input arc expression on the
arc from place \figitem{CanCommit}. This means that binding $b_{1Y}$
is enabled and may occur. In fact, all the four bindings listed above
is enabled in the marking shown in Fig.~\ref{fig:commit}.

% ocurrence and more complicated evaluation

The tokens produced on output places when an transition occur in an
enabled binding is determined by evaluating the output arc expressions
of the transition in the given binding. Consider again the binding
element $b_{1Y}$. The output arc expression \smlcode{(w,vote)} will
evaluate to \smlcode{wrk(1),Yes} and this token will be added to place
\smlcode{Votes} to inform the coordinator that worker 1 votes yes to
committing the transaction. The arc expression on the arc from
\figitem{ReceiveCanCommit} to \figitem{Waiting} is an if-then-else
expression which in the binding $b_{1Y}$ will evaluate to the
multi-set $1`wrk(1)$ which will then be added to the tokens on place
\figitem{Waiting}. The if-then-else expression on the arc from
\figitem{ReceiveCanCommit} evaluates to the \smlcode{empty} multi-set
and hence no tokens will be added to place \figitem{Idle} in this
case. Figure~\ref{fig:receivecancommit} shows the marking resulting
from an occurrence of the $b_{1Y}$ binding.

\begin{figure}[h]
\centering
\includegraphics[scale=.5]{figures/ReceiveCanCommit.eps}
\caption{Current marking after \figitem{ReceiveCanCommit}.}
\label{fig:receivecancommit}
\end{figure}

The occurrence of the binding $b_{1N}$ representing that worker one
votes no would have the effect of removing a \smlcode{wrk(1)}-token
from \figitem{Idle}, and adding no tokens to place \figitem{Waiting}
and adding one \smlcode{wrk(1)}-token to place \figitem{Idle}. This
models the fact that if a worker votes no to committing the
transaction, then it goes back to idle; whereas if it votes yes, then
it will go to waiting to be informed about whether the transaction is
to be committed or not.  Recall that this is a distributed system and
hence a worker cannot (without exchanging messages) know what other
workers have voted.  Above we have considered relatively simple arc
expression but the arc expression of a transition can be any
expression that can be written in Standard ML as long as it has a type
that matches the corresponding place. In particular, arc expressions
may apply functions including higher-order functions.

All the four bindings listed for transition \figitem{ReceiveCanCommit}
were enabled in the marking shown in
Fig.~\ref{fig:worker}. Furthermore, enabled bindings may be
\concept{concurrently enabled} if each binding can get its required
multi-set of tokens from each input place independently of the other
enabled bindings in the set. As an example, the two bindings $b_{1Y}$
and $b_{2Y}$ are concurrent enabled since each binding can gets its
token from the input places without competing with each other. This
reflects that the workers are executing concurrently and may
simultaneously send a vote back to the coordinator. In contrast, the two
bindings $b_{1Y}$ and $b_{1N}$ are not concurrent. These two bindings
are in \concept{conflict} because they each need, e.g., the single
\smlcode{wrk(1)}-token on \figitem{Idle} (and in fact also the single
\smlcode{wrk(1)}-token on place \figitem{CanCommit}. The notion of
concurrency and conflict of binding extends to also span to bindings
of different transitions. A fundamental property of a set of enabled
bindings that CPNs inherit from Petri nets is that theses can be
executed in any interleaved order and the resulting marking will be
the same independently of the interleaved execution considered. 

\begin{itemize}

\item HERE WE COULD MAKE A FURTHER LINK TO PT-NETS BY SHOWING THE
  FRAGMENT CORRESPONDING TO RECEIVECANCOMMIT. THAT WOULD ILLUSTRATE
  UNFOLDING OF TRANSITIONS.

\item HERE WE COULD BRIEFLY MENTION GUARDS. 

\item WE COULD BRIEFLY TALK ABOUT MODELING TIME IF WE SKIP HAVING
  A SECTION ON THIS.

\end{itemize}


%\section{An Example CPN Model}

\begin{figure*}[t]
\centering
\includegraphics[width=15cm]{figures/Commit.eps}
\caption{The top-level Commit module of the CPN model.}
\end{figure*}

\begin{figure}[t]
\centering
\includegraphics[width=\columnwidth]{figures/Coordinator.eps}
\caption{The Coordinator module.}
\end{figure}

\begin{figure}[t]
\centering
\includegraphics[width=\columnwidth]{figures/CollectVotes.eps}
\caption{The CollectVotes module.}
\end{figure}

\begin{figure}[t]
\centering
\includegraphics[width=\columnwidth]{figures/Worker.eps}
\caption{The Worker module.}
\end{figure}

\begin{figure}
\begin{verbatim}
colset INT = int;
var i : INT;

colset Coordinator = with c;
colset CoordinatorxInt = product Coordinator * INT;

colset Worker = index wrk with  1..2;
var w: Worker;

colset WorkerList = list Worker;
var workers : WorkerList;

colset CoordinatorxWorkers = product Coordinator * WorkerList;

colset Vote = with Yes | No;
var vote : Vote;

colset WorkerxVotes = product Worker * Vote;
colset WorkerxVote = list WorkerxVotes;
var votes : WorkerxVote;

colset Decision = with abort | commit;
var decision : Decision;

colset WorkerxDecision = product Worker * Decision;
\end{verbatim}
\caption{Colour sets and variable declarations.}
\end{figure}

\begin{figure}
\begin{verbatim}
val Workers = Worker.all();

fun AddVote ((w,vote),votes) = 
    sort_ms WorkerxVotes.lt ((w,vote)::votes);

fun YesWorkers votes = 
    List.map (fn (w,_) => w) 
      (List.filter (fn (w,Yes) => true 
                     | (w,No) => false) votes);

fun InformWorkers votes = 
    if (List.all (fn (_,Yes) => true 
                   | _ => false) votes)
                                            
    then List.map (fn w => (w,commit)) (YesWorkers votes)
    else List.map (fn w => (w,abort)) (YesWorkers votes);

fun All votes = List.length votes = List.length (Worker.all());
\end{verbatim}
\caption{Values and function definitions.}
\end{figure}


\section{Timed CPNs}

\section{Analysis and Validation}


\section{CPN Tools and Applications}

The construction and analysis of CPN models have been supported by two
generations of graphical computer tools. The first generation was the
Design/CPN tool \cite{jensen:cpnmanual} which was developed starting
in the mid 80'es at Meta Software Corp. and later by the CPN Group at
the Aarhus University. This was followed by CPN Tools
\cite{cpntoolsweb} that has been developed since 2000 first by the CPN
Group at Aarhus University and since 2009? by the XX group at the
Technical University of Eindhoven. CPN Tools supports the editing and
construction of CPN models, interactive and automatic simulation,
state space-based model checking, and simulation-based performance
analysis. Both Design/CPN and CPN Tools has been widely distributed
tools and they have been applied for modelling and validation in a
broad range of application domains. Below we provide some pointer to
some selected applications within typical application domains. A more
comprehensive list of example applications and domains can be found
via \cite{cpnuse}.

\begin{figure*}[b]
\centering
\includegraphics[width=\textwidth]{figures/cpntools.png}
\caption{Two-phase Commit Protocol in CPN Tools}
\label{fig:cpntools}
\end{figure*}


Figure~\ref{fig:cpntools} provides a screenshot of CPN Tools. The user
of CPN Tools works directly with the graphical representation of the
CPN model. The graphical user interface (GUI) of CPN Tools has no
conventional menu bars and pull-down menus, but is based on
interaction techniques, such as \concept{tool palettes} and
\concept{marking menus}. The rectangular area to the left is an
\concept{index}. It includes the \figitem{Tool box}, which is
available for the user to manipulate the declarations and modules that
constitute the CPN model. The \figitem{Tool box} includes tools for
creating, copying, and cloning the basic elements of CP-nets. It also
contains a wide selection of tools to manipulate the graphical layout
and the appearance of the objects in the CPN model. The latter set of
tools is very important in order to be able to create readable and
graphically appealing CPN models. The remaining part of the screen is
the \concept{workspace}, which in this case contains four
\concept{binders} (the rectangular windows) and a circular pop-up
menu.

Each binder holds a number of items which can be accessed by clicking
the tabs at the top of the binder (only one item is visible at a
time). There are two kinds of binders. One kind contains the elements
of the CPN model, i.e., the modules and declarations. The other kind
contains the tools which the user applies to construct and manipulate
CPN models. The tools in a tool palette can be picked up with the
mouse cursor and applied. In the example shown, one binder contains
three modules named \figitem{Protocol}, \figitem{Sender}, and
\figitem{Receiver}, while another binder contains a single module,
named \figitem{Network}, together with the declaration of the colour
set \smlcode{NOxDATA}. The two remaining binders contain four
different tool palettes to \figitem{Create} elements, change their
\figitem{Style}, perform \figitem{Simulations}, and construct
\figitem{State spaces}.

Items can be dragged from the index to the binders, and from one
binder to another binder of the same kind. It is possible to position
the same item in two different binders, for example, to view a module
using two different zoom factors. A circular marking menu has been
popped up on top of the bottom left binder. Marking menus are
contextual menus that make it possible to select among the operations
possible on a given object. In the case of
Fig.~\ref{fig:cpntoolssnapshot}, the marking menu gives the operations
that can be performed on a port place object.

% syntax check and code generation

CPN Tools performs syntax and type checking, and error messages are
provided to the user in a contextual manner next to the object causing
the error. The syntax check and code generation are incremental and
are performed in parallel with editing. This means that it is possible
to execute parts of a CPN model even if the model is not complete, and
that when parts of a CPN model are modified, a syntax check and code
generation are performed only on the elements that depend on the parts
that were modified. The main outcome of the code generation step is
the \concept{simulation code}. The simulation code contains the
functions for inferring the set of enabled events in a given state of
the CPN model, and for computing the state resulting from the
occurrence (execution) of an enabled event in a given state.

% simulation

CPN Tools supports two types of simulation: interactive and
automatic. In an interactive simulation, the user is in complete
control and determines the individual steps in the simulation, by
selecting between the enabled events in the current state. CPN Tools
shows the effect of executing a selected step in the graphical
representation of the CPN model. In an automatic simulation the user
specifies the number of steps that are to be executed and/or sets a
number of stop criteria and breakpoints. The simulator then
automatically executes the model without user interaction by making
random choices between the enabled events in the states
encountered. Only the resulting state is shown in the GUI. A
\concept{simulation report} can be saved, containing a specification
of the steps that occurred during an automatic simulation. The
simulator of CPN Tools exploits a number of advanced data structures
for efficient simulation of large hierarchical CPN models. The
simulator exploits the locality property of Petri nets to ensure that
the number of steps executed per second in a simulation is independent
of the size of the CPN model. This guarantees that simulation scales
to large CPN models.

TALK/REFER TO THINGS ABOUT INDUSTRIAL APPLICATIONS IN THE END

\paragraph{Embedded Systems} Dalcotech or B and O

\paragraph{Process Scheduling} COAST

\paragraph{Internet Protocols} ERDP

\paragraph{Mobile Phone Software} NOKIA

\paragraph{Capacity Planning} HP



\bibliographystyle{abbrv}
\bibliography{sigproc}  % sigproc.bib is the name of the Bibliography in this case
\end{document}
