\section{Coloured Petri Nets}

To present the concepts of CPNs, we use a CPN model of a distributed
two-phase commit transaction system. We use the example to give an
informal introduction to CPNs. The reader interested in the formal
definition of CPNs is referred to \cite{X}.

% modules - substitution transitions

The CPN model of the two-phase commit system is comprised of 4
\concept{modules} hierarchically organized into three
level. Figure~\ref{fig:commit} shows the top-level module which
consists of two \concept{substitution transitions} (drawn as
rectangles with double-lined borders) representing the
\figitem{Coordinator} and the \concept{Workers} in the system. Each of
the substitution transition has an associated \concept{submodule} that
model the behavior of the coordinator and the workers,
respectively. The name of an associated submodule is written in the
rectangular tag positioned at the bottom of each substitution
transition.

\begin{figure}[b]
\centering
\includegraphics[scale=.52]{figures/Commit.eps}
\caption{The top-level module of the CPN model.}
\label{fig:commit}
\end{figure}

% places - tokens - colours - and colour sets

The two substitution transitions are connected via directed
\concept{arcs} to the four places \figitem{CanCommit},
\figitem{Votes}, \figitem{Decision}, and \figitem{Acknowledge} (drawn
as ellipses). Places connected to substitution transitions are called
\concept{socket places} and are linked to \concept{port} places on the
associated submodules (to be presented shortly). The coordinator and
the workers interact by producing and consuming \concept{tokens} on
the places. These tokens carry data values and the type of tokens that
may reside on a place is determined by the \concept{color set} (type)
of the place (written in text below the
place). Figure~\ref{fig:coloursets} lists the definitions of the
color sets used as color set of the four places in
Fig.~\ref{fig:commit}. These color sets are defined using the CPN ML
programming language.

The \smlcode{Worker} color set is an indexed color set consisting of
the values \smlcode{wrk(1),wrk(2),...,wrk(W)} where the symbolic
constant \smlcode{W} is used to specify the number of worker processes
considered. This color set is used to model the identity of the
worker processes. The color set \smlcode{Vote} is an enumeration
color set containing the values \smlcode{Yes} and \smlcode{No}, and
is used to model that a worker by vote yes or no to commit the
transaction. The color set \smlcode{WorkerxVote} is a product type
containing pairs consisting of a worker and a vote. The color set
\smlcode{Decision} is an enumeration color set used to model whether
the coordinator decides to \smlcode{abort} or \smlcode{commit} the
transaction (only if all workers vote yes will the transaction be
committed). It should be noted that in addition to the type
constructors introduced above, CPN ML supports union, lists, and
record types. The variables \smlcode{w} and \smlcode{vote} declared in
Fig.~\ref{fig:coloursets} will be introduced later.

\begin{figure}[]
\begin{verbatim}
val W = 2;
colset Worker = index wrk with  1..W;
var w : Worker;

colset Vote        = with Yes | No;
var vote : Vote;
colset WorkerxVote = product Worker * Vote;

colset Decision        = with abort | commit;
colset WorkerxDecision = product Worker * Decision;
\end{verbatim}
\caption{Color sets definition used in Fig.~\ref{fig:commit}.}
\label{fig:coloursets}
\end{figure}


% marking - tokens initial marking - current marking

The state of a CPN model is called a \concept{marking}), and consists
of a distribution of \concept{tokens} on the places of the model. Each
place may hold a (possibly empty) \concept{multi-set} of tokens with
data values (colors) from the color set of the place. The
\concept{initial marking} of a place is specified above each place
(and omitted if the initial marking is the empty multi-set). For the
places in Fig.~\ref{fig:commit} all places are empty in the initial
marking. Initially, the \concept{current marking} of a CPN model
equals the initial marking. When a CPN model is executed occurrences of
enabled transitions consume and produce tokens on the places which
will change the \concept{current marking} of the CPN model.

 
% coordinator  - port places

The \figitem{Coordinator} module is shown in
Fig.~\ref{fig:coordinator}. This is the submodule associated with the
\figitem{Coordinator} submodule in Fig.~\ref{fig:commit}. The places
\figitem{CanCommit}, \figitem{Votes}, \figitem{Decision}, and
\figitem{Acknowledge} are \concept{port places} as indicated by the
rectangular \figitem{In} and \figitem{Out} tags positioned next to
them. These places are linked to the accordingly named places in the
top-level module (Fig.~\ref{fig:commit}) via a \concept{port-socket
  association} which implies that any tokens added/removed from a port
place by transitions in the \concept{Coordinator} module will also be
reflected in the marking of the associated socket place in the
top-level module. The places \figitem{CanCommit} and
\figitem{Decision} are \concept{output port places} which means that
the \figitem{Coordinator} module will only produce tokens on these
places. The places \figitem{Votes} and \figitem{Acknowledge} are
\concept{input port places} which means that the module will only
consume tokens from these places. It is also possible for a place to
be an input-output port place which means that the module may both
produce and consume tokens on (from) this place.

The places \figitem{Idle}, \figitem{WaitingVotes}, and
\figitem{WaitingAcknowledgement} are used to model the states that the
coordinator goes through when executing the two-phase commit
protocol. The places \figitem{Idle} and \figitem{WaitingVotes} have
the color set \smlcode{UNIT} containing just a single value
\smlcode{()} (denoted unit). Initially, the coordinator is in an idle
state as modeled by the initial marking of place
\figitem{Coordinator} which consist of a single token with the color
unit. In CPN ML this multi-set is written \smlcode{1`()} specifying
one (\smlcode{1}) occurrence of (\smlcode{`}) the unit color
(\smlcode{()}).  The number of tokens on a place in the current
marking is indicated with a small circle positioned next to a place,
and the detail of the color of the tokens are provided in an
associated text box. The indication of the current marking of a place
is omitted if currently the place contains no tokens.

The transitions \figitem{SendCanCommit}, \figitem{CollectVotes}, and
\figitem{ReceiveAcknowledge} model the event/actions that cause the
coordinator to change state. The coordinator will first send a can
commit message (transition \figitem{SendCanCommit}) to each worker
asking whether they can commit the transaction; then the coordinator
will collect the votes from all workers (transition
\figitem{CollectVotes}) and then send a decision to each worker
indicating whether the transaction is to be committed or not. Finally,
the coordinator will receive an acknowledgment from each worker that
they have received the decision (transition
\figitem{ReceiveAcknowledgements}).  It should be noted that
\figitem{CollectVotes} is a substitution transition which means that
the details of how the coordinator collects votes is modeled by the
associated \figitem{CollectVotes} submodule. This illustrates that it
is possible to mix the use of ordinary and substitution transitions
within a module.

\begin{figure}[]
\centering
\includegraphics[scale=.5]{figures/Coordinator.eps}
\caption{The \figitem{Coordinator} module.}
\label{fig:coordinator}
\end{figure}


% basic enabling and occurrences

In the current marking shown in Fig.~\ref{fig:coordinator} only the
transition \figitem{SendCanCommit} is enabled as indicated by the
thick border of that transition. The requirement for a transition to
be \concept{enabled} is determined from the \concept{arc expressions}
associated with the incoming arcs of the transition. In this case,
there is only a single incoming arc from place \figitem{Idle}
containing the expression \smlcode{()}. This expression specifies that
for \figitem{SendCanCommit} to be enabled, there must be at least one
\smlcode{()}-token present on \smlcode{Idle}. When the
\smlcode{SendCanCommit} transition \concept{occurs}, it will consume a
\smlcode{()}-token from place \smlcode{Idle} and it will produce
tokens on place connected to output arcs as determining by evaluating
the arc expressions on output arcs. In this case, the expression
\smlcode{()} on the arc to \smlcode{WaitingVotes} evaluates to a
single \smlcode{()}-token. The expression \smlcode{Worker.all ()} is a
function call when calls a function \smlcode{Worker.all} that takes a
unit value as argument and returns all colors of the color set
\smlcode{Worker}. This illustrates that arc expression may also make
use of function to calculate the tokens to be added (removed) from
places. In particular this means that complex data manipulation can be
performed without intruding intermediate steps in the model itself.

Figure~\ref{fig:sendcancommit} shows the marking of the surrounding
places of transition \figitem{SendCanCommit} after the occurrence of
\figitem{SendCanCommit}. It can be seen that the place
\smlcode{CanCommit} contains two tokens - one of each worker in the
system - representing messages going to the two worker processes. The
coordinator has now entered a state in which it is waiting to collect
the votes from the worker processes.

\begin{figure}[h]
\centering
\includegraphics[scale=.5]{figures/SendCanCommit.eps}
\caption{Current marking after \figitem{SendCanCommit}.}
\label{fig:sendcancommit}
\end{figure}

% comparison to low-level nets

By setting the symbolic constant \smlcode{W} (see
Fig.~\ref{fig:coloursets}) we can easily configure to model to handle,
e.g., five workers. With ordinary Petri nets we would have had to
create a copy of the \figitem{CanCommit} place for each worker. In
particular, we would have to make changes to the net structure
(places, transitions, arcs) when changing the number of workers. This
shows that CPNs provides a means for easily creating parameterizable
models and also that it enables more compact modeling as we only need
a single instance of the \figitem{CanCommit} place in order to
accommodate any finite number of workers. 

\begin{itemize}
\item WE COULD EABORATE MORE ON THIS BY DRAWING THE CORRESPONDING
  PT-NET FOR THE CANCOMMIT TRANSITION. THAT WOULD ILLUSTRATE UNFOLDING
  OF PLACES (see comment at the end of this section that proposes
  illustration of unfolding transition. Perhaps it is a good approach
  to split the introduction of unfolding in two steps. Then at the end
  of the section we can mention that any CPNs can be unfolder to a
  possibly infinite behaviorally equivalent PT-nets. The two example
  that we have in this section would then nicely illustrate how. It
  would also be possible to factor this into a seperate section in
  order to avoid that this section becomes overlong in comparison with
  the other section.  
\end{itemize}

%\com{Maybe we could have shown
%  the corresponding fragment as a Place/Transition net? In this case we need more places - when we get to the worker we could also need to duplicate transition as per bindings}

% worker process - and binding and binding elements.

Figure~\ref{fig:worker} shows the \figitem{Worker} module which is the
submodule of the \figitem{Workers} substitution transition in
Fig.~\ref{fig:commit}. The places \figitem{CanCommit},
\figitem{Votes}, \figitem{Decision}, and \figitem{Acknowledge}
constitute the port places of this module and are linked to the
accordingly named socket places in Fig.~\ref{fig:commit}. The places
\figitem{Idle} and \figitem{Waiting} models the two main states of
worker processes. Each of these places have the color set
\smlcode{Worker} and the idea is that when there is a token with
color \smlcode{wrk(i)} on, e.g., the place \figitem{Idle}, then this
represent that the i'th worker is in state idle. This makes it
possible to model the state of all workers in a compact manner within
a single module without having to have a place for each worker or a
module instance for each worker. Initially, all workers are in the
idle state as represented by corresponding tokens on place
\figitem{Idle} in the initial marking. The transition
\figitem{ReceiveCanCommit} models the reception of can commit messages
from the coordinator and the sending of a vote. The transition
\figitem{ReceiveDecision} models the reception of a decision message
from the coordinator and the sending of an acknowledgment.

\begin{figure}[]
\centering
\includegraphics[scale=.5]{figures/Worker.eps}
\caption{The \figitem{Worker} module.}
\label{fig:worker}
\end{figure}

The current marking of place \figitem{CanCommit} is \smlcode{1`wrk(1)
  ++ 1`wrk(2)} corresponding a marking where the coordinator has sent
a can commit message to each worker. The thick border of transition
\figitem{ReceiveCanCommit} indicates that this transition is enabled
in the current marking. The arc expressions on the surrounding arcs
of the \figitem{ReceiveCanCommit} transition are more complex than the
arc expressions of the \figitem{SendCanCommit} transition in the
\figitem{Coordinator} module considered earlier in that they contain
the \smlcode{free variables} \smlcode{w} and \smlcode{vote} defined in
Fig.~\ref{fig:coloursets}. This means that in order to talk about the
enabling and occurrence of transition \smlcode{ReceiveCanCommit}, we
need to assign values to these variables in order to evaluate the
input and output arc expressions. This is done by creating a
\concept{binding} which associates a value to each of the free
variables occurring in the arc expression of the transition. Bindings
can be considered different modes in which a transition may occur.  As
\smlcode{w} is of type \smlcode{Worker} and \smlcode{vote} is of type
\smlcode{Decision}, this gives the following four possible bindings
reflecting that each of the two workers may vote yes or no to
committing the transactions:

\begin{eqnarray*}
b_{1Y} & = & \langle \smlcode{w} = \smlcode{wrk(1)}, \smlcode{vote} = \smlcode{Yes} \rangle \\
b_{1N} & = & \langle \smlcode{w} = \smlcode{wrk(1)}, \smlcode{vote} = \smlcode{No} \rangle \\
b_{2Y} & = & \langle \smlcode{w} = \smlcode{wrk(2)}, \smlcode{vote} = \smlcode{Yes} \rangle \\
b_{2N} & = & \langle \smlcode{w} = \smlcode{wrk(2)}, \smlcode{vote} = \smlcode{No} \rangle
\end{eqnarray*}

A binding of a transition is enabled if evaluating each input arc
expression in the binding results in a multi-set of tokens which is a
subset of the multi-set of tokens present on the corresponding input
place. For an example, consider the binding $b_{1Y}$. Evaluating the
input arc expression \smlcode{w} on the input arc from \smlcode{Idle}
results in the multi-set containing a single token with the color
\smlcode{wrk(1)} which is contained in the multi-set of tokens present
on place \figitem{Idle} in the marking depicted in
Fig.~\ref{fig:worker}. Similarly for the input arc expression on the
arc from place \figitem{CanCommit}. This means that binding $b_{1Y}$
is enabled and may occur. In fact, all the four bindings listed above
is enabled in the marking shown in Fig.~\ref{fig:commit}.

% ocurrence and more complicated evaluation

The tokens produced on output places when an transition occur in an
enabled binding is determined by evaluating the output arc expressions
of the transition in the given binding. Consider again the binding
element $b_{1Y}$. The output arc expression \smlcode{(w,vote)} will
evaluate to \smlcode{wrk(1),Yes} and this token will be added to place
\smlcode{Votes} to inform the coordinator that worker 1 votes yes to
committing the transaction. The arc expression on the arc from
\figitem{ReceiveCanCommit} to \figitem{Waiting} is an if-then-else
expression which in the binding $b_{1Y}$ will evaluate to the
multi-set $1`wrk(1)$ which will then be added to the tokens on place
\figitem{Waiting}. The if-then-else expression on the arc from
\figitem{ReceiveCanCommit} evaluates to the \smlcode{empty} multi-set
and hence no tokens will be added to place \figitem{Idle} in this
case. Figure~\ref{fig:receivecancommit} shows the marking resulting
from an occurrence of the $b_{1Y}$ binding.

\begin{figure}[h]
\centering
\includegraphics[scale=.5]{figures/ReceiveCanCommit.eps}
\caption{Current marking after \figitem{ReceiveCanCommit}.}
\label{fig:receivecancommit}
\end{figure}

The occurrence of the binding $b_{1N}$ representing that worker one
votes no would have the effect of removing a \smlcode{wrk(1)}-token
from \figitem{Idle}, and adding no tokens to place \figitem{Waiting}
and adding one \smlcode{wrk(1)}-token to place \figitem{Idle}. This
models the fact that if a worker votes no to committing the
transaction, then it goes back to idle; whereas if it votes yes, then
it will go to waiting to be informed about whether the transaction is
to be committed or not.  Recall that this is a distributed system and
hence a worker cannot (without exchanging messages) know what other
workers have voted.  Above we have considered relatively simple arc
expression but the arc expression of a transition can be any
expression that can be written in Standard ML as long as it has a type
that matches the corresponding place. In particular, arc expressions
may apply functions including higher-order functions.

All the four bindings listed for transition \figitem{ReceiveCanCommit}
were enabled in the marking shown in
Fig.~\ref{fig:worker}. Furthermore, enabled bindings may be
\concept{concurrently enabled} if each binding can get its required
multi-set of tokens from each input place independently of the other
enabled bindings in the set. As an example, the two bindings $b_{1Y}$
and $b_{2Y}$ are concurrent enabled since each binding can gets its
token from the input places without competing with each other. This
reflects that the workers are executing concurrently and may
simultaneously send a vote back to the coordinator. In contrast, the two
bindings $b_{1Y}$ and $b_{1N}$ are not concurrent. These two bindings
are in \concept{conflict} because they each need, e.g., the single
\smlcode{wrk(1)}-token on \figitem{Idle} (and in fact also the single
\smlcode{wrk(1)}-token on place \figitem{CanCommit}. The notion of
concurrency and conflict of binding extends to also span to bindings
of different transitions. A fundamental property of a set of enabled
bindings that CPNs inherit from Petri nets is that theses can be
executed in any interleaved order and the resulting marking will be
the same independently of the interleaved execution considered. 

\begin{itemize}

\item HERE WE COULD MAKE A FURTHER LINK TO PT-NETS BY SHOWING THE
  FRAGMENT CORRESPONDING TO RECEIVECANCOMMIT. THAT WOULD ILLUSTRATE
  UNFOLDING OF TRANSITIONS.

\item HERE WE COULD BRIEFLY MENTION GUARDS. 

\item WE COULD BRIEFLY TALK ABOUT MODELING TIME IF WE SKIP HAVING
  A SECTION ON THIS.

\end{itemize}
