\section{Coloured Petri Nets}
\label{sect:language}

% modules - substitution transitions

To give an informal introduction to the concepts of CPNs, we use a CPN
model of a distributed two-phase commit transaction system.\footnote{The full details of the CPN model presented in this paper are available via \texttt{http://goo.gl/HY4ZNQ}}.  The reader
interested in the formal definition of CPNs is referred to
\cite{newcpnbook}. The CPN model of the two-phase commit system is
comprised of four \concept{modules} hierarchically organized into
three levels. Figure~\ref{fig:commit} shows the top-level module which
consists of two \concept{substitution transitions} (drawn as
rectangles with double-lined borders) representing the
\figitem{Coordinator} and the \figitem{Workers} in the system. Each of
the substitution transitions has an associated \concept{submodule}
that models the detailed behavior of the coordinator and the workers,
respectively. The name of the submodule is written in the rectangular
tag positioned at the bottom of each substitution transition.

\begin{figure}[b]
\centering
\includegraphics[scale=.5]{figures/Commit.eps}
\caption{The top-level module of the CPN model.}
\label{fig:commit}
\end{figure}

% places - tokens - colours - and colour sets

The two substitution transitions are connected via directed
\concept{arcs} to the four places \figitem{CanCommit},
\figitem{Votes}, \figitem{Decision}, and \figitem{Acknowledge} (drawn
as ellipses). Places connected to substitution transitions are called
\concept{socket places} and are linked to \concept{port} places in the
associated submodules (to be presented shortly). The coordinator and
the workers interact by producing and consuming \concept{tokens} on
the places. These tokens carry data values and the type of tokens that
may reside on a place is determined by the \concept{type} of the place
(written in text below the place). For historical reasons the types of
places are called \concept{color sets}. Figure~\ref{fig:coloursets}
lists the definitions of the color sets used for the four places in
Fig.~\ref{fig:commit}. These color sets are defined using the CPN ML
programming language which is based on the functional language
Standard ML \cite{sml}.

The symbolic constant \smlcode{W} is used to specify the number of
worker processes considered. The \smlcode{Worker} color set is an
indexed type consisting of the values
\smlcode{wrk(1),wrk(2) .. wrk(W)} used to model the identity of the
worker processes. The color set \smlcode{Vote} is an enumeration type
containing the values \smlcode{Yes} and \smlcode{No}, and is used to
model that a worker may vote \smlcode{Yes} or \smlcode{No} to commit
the transaction. The color set \smlcode{WorkerxVote} is a product type
containing pairs consisting of a worker and its vote. The color set
\smlcode{Decision} is an enumeration type used to model whether the
coordinator decides to \smlcode{Abort} or \smlcode{Commit} the
transaction (only if all workers vote yes will the transaction be
committed). It should be noted that in addition to the type
constructors introduced above, CPN ML supports union, lists, and
record types. The variables \smlcode{w} and \smlcode{vote} declared in
Fig.~\ref{fig:coloursets} will be introduced later.

\begin{figure}[]
\begin{verbatim}
val W = 2;

colset Worker = index wrk with  1..W;
colset Vote = with Yes | No;
colset WorkerxVote = product Worker * Vote;

colset Decision = with Abort | Commit;
colset WorkerxDecision = product Worker * Decision;

var w : Worker;
var vote : Vote;
\end{verbatim}
\caption{Definition of color sets used in Fig.~\ref{fig:commit}.}
\label{fig:coloursets}
\end{figure}


% marking - tokens initial marking - current marking

The state of a CPN model is called a \concept{marking}, and consists
of a distribution of \concept{tokens} on the places of the model. Each
place may hold a (possibly empty) \concept{multi-set} of tokens with
data values (colors) from the color set of the place. The
\concept{initial marking} of a place is specified above each place
(and omitted if the initial marking is the empty multi-set). For the
places in Fig.~\ref{fig:commit}, all places are empty in the initial
marking. Initially, the \concept{current marking} of a CPN model
equals the initial marking. When a CPN model is executed
\concept{occurrences} of \concept{enabled transitions} consume and
produce tokens on the places changing the current marking of the CPN model.

 
% coordinator  - port places

The \figitem{Coordinator} module is shown in
Fig.~\ref{fig:coordinator}. This is the submodule associated with the
\figitem{Coordinator} substitution transition in
Fig.~\ref{fig:commit}. The places \figitem{CanCommit},
\figitem{Votes}, \figitem{Decision}, and \figitem{Acknowledge} are
\concept{port places} as indicated by the rectangular \figitem{In} and
\figitem{Out} tags positioned next to them. These places are linked to
the accordingly named socket places in the top-level module
(Fig.~\ref{fig:commit}) via a \concept{port-socket association} which
implies that any tokens added (removed) from a port place by
transitions in the \figitem{Coordinator} module will also be added
(removed) in the marking of the associated socket place in the
top-level module. The places \figitem{Votes} and \figitem{Acknowledge}
are \concept{input port places} which means that the
\figitem{Coordinator} module will only consume tokens from these
places.  The places \figitem{CanCommit} and \figitem{Decision} are
\concept{output port places} which means that the
\figitem{Coordinator} module will only produce tokens on these
places. It is also possible for a place to be an \concept{input-output
  port place} which means that the module may both consume and produce
tokens on this place.

\begin{figure}[]
\centering
\includegraphics[scale=.45]{figures/Coordinator.eps}
\caption{The \figitem{Coordinator} module.}
\label{fig:coordinator}
\end{figure}



The places \figitem{Idle}, \figitem{WaitingVotes}, and
\figitem{WaitingAcknowledgements} are used to model the states of the
coordinator when executing the two-phase commit protocol. The places
\figitem{Idle} and \figitem{WaitingVotes} have the color set
\smlcode{UNIT} containing just a single value \smlcode{()} (denoted
unit). Initially, the coordinator is in an idle state as modeled by
the initial marking of place \figitem{Idle} which consists of a single
token with the color unit. In CPN ML this multi-set is written
\smlcode{1`()} specifying one (\smlcode{1}) occurrence of
(\smlcode{`}) the unit color (\smlcode{()}).  The number of tokens on
a place in the current marking is indicated with a small circle
positioned next to a place, and the detail of the color of the tokens
are provided in an associated text box. The indication of the current
marking of a place is omitted if currently the place contains no
tokens.

The transitions \figitem{SendCanCommit}, \figitem{CollectVotes}, and
\figitem{Receive\linebreak[4]Acknowledgements} model the
events/actions that cause the coordinator to change state. The
coordinator will first send a can commit message (transition
\figitem{SendCanCommit}) to each worker asking whether they can commit
the transaction. Then the coordinator will collect the votes from all
workers and send a decision to the workers that voted yes indicating
whether the transaction is to be committed or not (substitution
transition \figitem{CollectVotes}). Finally, the coordinator will
receive an acknowledgment from each worker that voted yes, confirming
that they have received the decision (transition
\figitem{ReceiveAcknowledgements}).  It should be noted that
\figitem{CollectVotes} is a substitution transition which means that
the details of how the coordinator collects votes is modeled by the
associated \figitem{CollectVotes} submodule. This illustrates the
mixed use of ordinary and substitution transitions within a module. We
omit the details of \figitem{CollectVotes} in this paper.

% basic enabling and occurrences

In the current marking shown in Fig.~\ref{fig:coordinator} only the
transition \figitem{SendCanCommit} is enabled as indicated by the
thick border of that transition. The requirement for a transition to
be \concept{enabled} is determined from the \concept{arc expressions}
associated with the incoming arcs of the transition. In this case,
there is only a single incoming arc from place \figitem{Idle}
containing the expression \smlcode{()}. This expression specifies that
for \figitem{SendCanCommit} to be enabled, there must be at least one
\smlcode{()}-token present on \figitem{Idle}. When the
\figitem{SendCanCommit} transition \concept{occurs}, it will consume a
\smlcode{()}-token from place \smlcode{Idle} and it will produce
tokens on places connected to output arcs as determined by
\concept{evaluating} the arc expressions on output arcs. In this case,
the expression \smlcode{()} on the arc to \smlcode{WaitingVotes}
evaluates to a single \smlcode{()}-token. The expression
\smlcode{Worker.all()} is a call to the function \smlcode{Worker.all}
that takes a unit value (\smlcode{()}) as parameter and returns all
colors of the color set \smlcode{Worker}. This illustrates how complex
calculations (e.g., of multi-sets of tokens or involving tokens with
complex data values) can be encapsulated within function calls, and
how several tokens can be added/removed in a single step (transition
occurrence) without introducing intermediate states (markings).

\ignore{
This illustrates that arc expression may also make
use of function to calculate the tokens to be added (removed) from
places. In particular this means that complex data manipulation can be
performed without intruding intermediate steps in the model itself.
}

Figure~\ref{fig:sendcancommit} shows the marking of the surrounding
places of transition \figitem{SendCanCommit} after the occurrence of
\figitem{SendCanCommit}. It can be seen that the place
\figitem{CanCommit} contains two tokens (one of each worker in the
system) representing messages going to the two worker processes. The
coordinator has now entered a state in which it is waiting to collect
the votes from the worker processes.

\begin{figure}[b]
\centering
\includegraphics[scale=.45]{figures/SendCanCommit.eps}
\caption{Current marking after \figitem{SendCanCommit}.}
\label{fig:sendcancommit}
\end{figure}

% comparison to low-level nets


%\com{Maybe we could have shown
%  the corresponding fragment as a Place/Transition net? In this case we need more places - when we get to the worker we could also need to duplicate transition as per bindings}

% worker process - and binding and binding elements.

Figure~\ref{fig:worker} shows the \figitem{Workers} module which is
the submodule of the \figitem{Workers} substitution transition in
Fig.~\ref{fig:commit}. The places \figitem{CanCommit},
\figitem{Votes}, \figitem{Decision}, and \figitem{Acknowledge}
constitute the port places of this module and are linked to the
accordingly named socket places in Fig.~\ref{fig:commit}. The places
\figitem{Idle} and \figitem{WaitingDecision} model the two states of
worker processes. Each of these places have the color set
\smlcode{Worker} and the idea is that when there is a token with color
\smlcode{wrk(i)} on, e.g., the place \figitem{Idle}, then this models
that the i'th worker is in state idle. This makes it possible to model
the state of all workers in a compact manner within a single module
without having to have a place for each worker or a module instance
for each worker. Initially, all workers are in the idle state as
represented by corresponding tokens on place \figitem{Idle} in the
initial marking. The transition \figitem{ReceiveCanCommit} models the
reception of can-commit messages from the coordinator and the sending
of a vote. The transition \figitem{ReceiveDecision} models the
reception of a decision message from the coordinator and the sending
of an acknowledgment.

\begin{figure}[]
\centering
\includegraphics[scale=.45]{figures/Workers.eps}
\caption{The \figitem{Workers} module.}
\label{fig:worker}
\end{figure}

The current marking of place \figitem{CanCommit} in
Fig.~\ref{fig:worker} is \smlcode{1`wrk(1) ++ 1`wrk(2)} modeling a
marking where the coordinator has sent a can-commit message to each
worker. The thick border of transition \figitem{ReceiveCanCommit}
indicates that this transition is enabled in the current marking. The
arc expressions on the surrounding arcs of the
\figitem{ReceiveCanCommit} transition are more complex than the arc
expressions of the \figitem{SendCanCommit} transition in the
\figitem{Coordinator} module considered earlier in that they contain
the \concept{free variables} \smlcode{w} and \smlcode{vote} defined in
Fig.~\ref{fig:coloursets}. This means that in order to talk about the
enabling and occurrence of transition \figitem{ReceiveCanCommit}, we
need to bind (assign) values to these variables in order to evaluate
the input and output arc expressions. This is done by creating a
\concept{binding} which associates a value to each of the free
variables occurring in the arc expressions of the transition. Bindings
can be considered different modes in which a transition may occur.  As
\smlcode{w} is of type \smlcode{Worker} and \smlcode{vote} is of type
\smlcode{Decision}, this gives the bindings listed in
Fig.~\ref{fig:bindings} reflecting that each of the two workers may
vote \smlcode{Yes} or \smlcode{No} to committing the transactions.

\begin{figure}[]
\centering
\begin{eqnarray*}
b_{1Y} & = & \langle \smlcode{w} = \smlcode{wrk(1)}, \smlcode{vote} = \smlcode{Yes} \rangle \\
b_{1N} & = & \langle \smlcode{w} = \smlcode{wrk(1)}, \smlcode{vote} = \smlcode{No} \rangle \\
b_{2Y} & = & \langle \smlcode{w} = \smlcode{wrk(2)}, \smlcode{vote} = \smlcode{Yes} \rangle \\
b_{2N} & = & \langle \smlcode{w} = \smlcode{wrk(2)}, \smlcode{vote} = \smlcode{No} \rangle
\end{eqnarray*}
\caption{Bindings of transition \figitem{ReceiveCanCommit}.}
\label{fig:bindings}
\end{figure}

A binding of a transition is enabled if evaluating each input arc
expression in the binding results in a multi-set of tokens which is a
subset of the multi-set of tokens present on the corresponding input
place. For an example, consider the binding $b_{1Y}$. Evaluating the
input arc expression \smlcode{w} on the input arc from \figitem{Idle}
results in the multi-set containing a single token with the color
\smlcode{wrk(1)} which is contained in the multi-set of tokens present
on place \figitem{Idle} in the marking depicted in
Fig.~\ref{fig:worker}. Similarly for the input arc expression on the
arc from place \figitem{CanCommit}. This means that binding $b_{1Y}$
is enabled and may occur. In fact, all four bindings listed in
Fig.~\ref{fig:bindings} is enabled in the marking shown in
Fig.~\ref{fig:worker}.

% ocurrence and more complicated evaluation

The tokens produced on output places when a transition occurs in an
enabled binding is determined by evaluating the output arc expressions
of the transition in the given binding. Consider again the binding
element $b_{1Y}$. The output arc expression \smlcode{(w,vote)}
evaluates to \smlcode{(wrk(1),Yes)} and this token will be added to
place \figitem{Votes} to inform the coordinator that worker one votes
yes to committing the transaction. The arc expression on the arc from
\figitem{ReceiveCanCommit} to \figitem{WaitingDecision} is an
if-then-else expression which in the binding $b_{1Y}$ evaluates to the
multi-set \smlcode{1`wrk(1)} which will be added to the tokens on
place \figitem{WaitingDecision}. The if-then-else expression on the
arc from \figitem{ReceiveCanCommit} to \figitem{Idle} evaluates to the
\smlcode{empty} multi-set and hence no tokens will be added to place
\figitem{Idle} in this case. Figure~\ref{fig:receivecancommit} shows
the marking resulting from an occurrence of the $b_{1Y}$ binding.

\begin{figure}[b]
\centering
\includegraphics[scale=.45]{figures/ReceiveCanCommit.eps}
\caption{Current marking after \figitem{ReceiveCanCommit}.}
\label{fig:receivecancommit}
\end{figure}

The occurrence of the binding $b_{1N}$ representing that worker one
votes no would have the effect of removing a \smlcode{wrk(1)}-token
from \figitem{Idle}, adding a \smlcode{(wrk(1),No)}-token to
\figitem{Votes}, adding no tokens to place \figitem{WaitingDecision},
and adding a \smlcode{wrk(1)}-token to place \figitem{Idle}. This
models the fact that if a worker votes no to committing the
transaction, then it goes back to \figitem{Idle}; whereas if it votes
yes, then it will go to \figitem{WaitingDecision}.  As this is a
distributed system a worker cannot without exchanging messages know
the vote of another worker.

% concurrency
All four bindings listed for transition \figitem{ReceiveCanCommit}
were enabled in the marking shown in Fig.~\ref{fig:worker}. Enabled
bindings may be \concept{concurrently enabled} if each binding can get
its required multi-set of tokens from each input place independently
of the other enabled bindings in the set. As an example, the two
bindings $b_{1Y}$ and $b_{2Y}$ are concurrently enabled since each
binding can get its token from the input places without sharing with
each other. This reflects that the workers are executing concurrently
and may simultaneously send a vote back to the coordinator. In
contrast, the two bindings $b_{1Y}$ and $b_{1N}$ are not concurrently
enabled. These two bindings are in \concept{conflict} because they
each need the single \smlcode{wrk(1)}-token on \figitem{Idle} (and in
fact also the single \smlcode{wrk(1)}-token on place
\figitem{CanCommit}). The notion of concurrency and conflicts of binding
extends to bindings of different transitions. A fundamental property
of a set of concurrently enabled bindings that CPNs inherits from Petri
nets is that these can be executed in any interleaved order and the
resulting marking will be the same independently of the interleaved
execution considered.

% a few comments about other constructed in the language
 Above relatively simple arc expressions were used. In general, arc
 expressions can be any expression that can be written in Standard ML
 as long as they have types that matches the corresponding places. In
 particular, arc expressions may apply functions including
 higher-order functions. CPNs also include a notion of time inspired
 by the work of van der Aalst \cite{aalst:93} that makes it possible
 to model the time taken by different activities. It is based on the
 introduction of a \concept{global clock} that represents the current
 model time and on attaching \concept{time stamps} to tokens in
 addition to the token colors. The time stamp of a token specifies the
 earliest model time at which the token can be removed by the
 occurrence of a transition, and delay inscriptions on the transitions
 and arcs are used to determine the timestamps on tokens produced by
 transitions. During model execution, the global model clock is always
 advanced to the earliest next time at which a transition becomes
 enabled, and the model stays at the current model time until no more
 transitions are enabled. The time concept provides the foundation for
 conducting simulation-based performance analysis of CPN models.

% In addition, it is also possible to associate
% \concept{guards} to transitions which are boolean expressions
% specifying additional constraints on the enabling of transition.

%i(see
% sidebar on Performance Analysis).

