\section{Tools and Applications}

The construction and analysis of CPN models have been supported by two
generations of graphical software tools: Design/CPN \cite{tacas97} and
CPN Tools \cite{cpn2003}. These tools have been instrumental to the
success of CPNs as they have enabled the practical use in a broad
range of domains such as distributed software systems, communication
protocols, embedded systems, and process- and workflow modeling. A
comprehensive list containing more than hundred papers describing
practical applications and domains is available \cite{cpnuse}. Both
Design/CPN and CPN Tools work direcetly on the high-level
representation of CPN models and do not perform unfolding to the
underlying PTN model (cf. Sect.~\ref{sect:unfolding}).

% Design/CPN and CPN Tools - history
The Design/CPN tool \cite{tacas97} was created at Meta Software,
Cambridge, Massachusetts, USA starting in 1988. The main architects
behind the tool were Jensen, Shapiro and Huber, and the implementation
was made together with an international group of people. The first
version of Design/CPN supported modeling, syntax cheek and interactive
simulation. The introduction of hierarchical CPNs supported by the
Design/CPN tool made a dramatic change to the practical use of Petri
nets. The new modeling language and its tool support were general and
powerful enough to eliminate the need of making ad-hoc extensions as
discussed in the introduction of this paper. A common platform for
practical modeling had been established and this was used by most
Petri net practitioners. The use of the platform was supported by a
three volume monograph on Colored Petri Nets published by Jensen in
1992-1997 \cite{jensen:cpnvols}.

Starting from year 2000 a second generation of tool support, called
CPN Tools, was designed and implemented at Aarhus University. The main
architects behind the new tool were Jensen, Christensen, and
Westergaard \cite{cpn2003}.  It was based on empirical studies of the
use of Design/CPN and much easier and efficient to use when
constructing CPN models. In 2010, CPN Tools had $10,000$ licenses in
150 countries. At that time the development and maintenance of the
tool set were transferred to the group of van der Aalst at the
Technical University of Eindhoven \cite{cpn2003}. New updates with improved
functionality are made at a regular basis.

% Illustration of CPN Tools.
CPN Tools supports the editing and construction of CPN models,
interactive and automatic simulation, state space-based model checking
(see Sidebar~II), and simulation-based performance analysis (see
Sidebar~III). CPN Tools is based on a much faster simulation engine
developed by Haagh, Hansen, and Mortensen \cite{mortensen:01}. With
this simulation engine, many models run more than thousand times
faster compared to Design/CPN allowing complex automatic simulations
to be executed within seconds instead of hours. 


\begin{figure*}[t]
\centering
\includegraphics[scale=.36]{figures/cpntoolsnew.png}
\caption{The two-phase commit CPN model in CPN Tools}
\label{fig:cpntools}
\end{figure*}

Figure~\ref{fig:cpntools} provides a screen-shot of CPN Tools with the
CPN model considered in this paper. The user works directly with the
graphical representation of the CPN model, and the user interface is
based on interaction techniques such as \concept{tool palettes} and
\concept{marking menus}. The rectangular area to the left is an
\concept{index} and includes the \figitem{Tool box}, which is
available for the user to manipulate the declarations and modules. The
remaining part of the screen is the \concept{workspace}, which in this
case contains two \concept{binders} (the rectangular windows) and a
circular pop-up menu. The binder to the left contains the
\figitem{Commit} module and the binder to the right contains the
\figitem{Coordinator} module. In addition, two tool palettes are
shown: one (\figitem{Sim}) containing the tools that can be used for
simulation of the model, and another (\figitem{Create}) containing the
tools for creating CPN model elements. A circular marking menu has
been popped up on top of a transition and gives the operations that
can be performed on a transition.




\ignore{

 contains three
modules named \figitem{Protocol}, \figitem{Sender}, and
\figitem{Receiver}, while another binder contains a single module,
named \figitem{Network}, together with the declaration of the colour
set \smlcode{NOxDATA}. The two remaining binders contain four
different tool palettes to \figitem{Create} elements, change their
\figitem{Style}, perform \figitem{Simulations}, and construct
\figitem{State spaces}.

There are two kinds of binders. One kind contains the
elements of the CPN model, i.e., the modules and declarations. The
other kind contains the tools which the user applies to construct and
manipulate CPN models. The tools in a tool palette can be picked up
with the mouse cursor and applied. In the example shown, one binder
contains three modules named \figitem{Protocol}, \figitem{Sender}, and
\figitem{Receiver}, while another binder contains a single module,
named \figitem{Network}, together with the declaration of the color
set \smlcode{NOxDATA}. The two remaining binders contain four
different tool palettes to \figitem{Create} elements, change their
\figitem{Style}, perform \figitem{Simulations}, and construct
\figitem{State spaces}.
}

% syntax check and code generation and simulation

CPN Tools performs syntax and type checking, and contextual error
messages are provided to the user. The syntax check and code generation are
incremental and are performed in parallel with editing. This means
that it is possible to execute parts of a CPN model even if the model
is not complete, and that when parts of a CPN model are modified, a
syntax check and code generation are performed only on the elements
that depend on the parts that were modified. CPN Tools supports two
types of simulation: interactive and automatic. In an interactive
simulation, the user is in complete control and determines the
individual steps in the simulation, by selecting between the enabled
events in the current state. CPN Tools shows the effect of executing a
selected step in the graphical representation of the CPN model. In an
automatic simulation the user specifies the number of steps that are
to be executed and/or sets a number of stop criteria and
breakpoints. The simulator then automatically executes the model
without user interaction by making random choices between the enabled
events in the states encountered. Only the resulting state is shown in
the GUI but information about the executed steps can be collected in
various ways. CPN Tools also includes support for domain-specific
graphical feedback from simulations developed by Westergaard and
Lassen \cite{britney}.

\ignore{
  The
simulator of CPN Tools exploits a number of advanced data structures
for efficient simulation of large hierarchical CPN models. The
simulator exploits the locality property of Petri nets to ensure that
the number of steps executed per second in a simulation is independent
of the size of the CPN model. This guarantees that simulation scales
to large CPN models.
}




