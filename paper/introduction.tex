\section{Introduction}

The vast majority of IT systems today can be characterized as
concurrent and distributed systems in that their operation inherently
relies on communication, synchronization, and resource sharing between
concurrently executing software components and applications. This is a
development that has been accelerated first with the pervasive
presence of the Internet as a communication infrastructure, and in
recent years by, e.g, cloud- and web-based services, mobile
applications, and multi-core computing architectures.

% main motivation and application domain of CPNs

The development of Coloured Petri Nets (CPNs) was initiated in the
early 80'es when distributed systems were becoming a major paradigm
for future computing systems. The goal of the CPN modeling language
was to develop a formally founded modeling language for concurrent
systems that would make it possible to formally analyze and validate
concurrent systems, and which from a modeling perspective would scale
to industrial systems. A main motivation behind the research into CPNs
(and many other formal modeling languages) was that the engineering of
correct concurrent systems is a challenging task due to their complex
behavior which may result in subtle bugs if not carefully designed. As
concurrent systems are becoming still more pervasive and critical to
society, formal techniques for concurrent systems were -- and still is
-- a highly relevant technology to support the engineering of reliable
concurrent systems. Many formal modelling languages for concurrent
systems have been developed, and examples of widely used languages
include Statecharts \cite{statecharts}, Timed Automata
\cite{timedautomata}, and Promela \cite{promela}.

% historical perspectic on development and roots

At its origin, CPNs build on Petri nets (see Sidebar~I on Petri nets),
which were introduced by Carl Adam Petri in his doctoral thesis
published in 1962 \cite{capetri:thesis} as a formalism for concurrency
and synchronization. The introduction of Petri nets by C.A. Petri was
far ahead of the time where distributed systems were invented and
computers started to have parallel processes. At that time, programs
and processing were considered to be sequential and
deterministic. Hence, it was extremely visionary of C.A. Petri to
predict the importance of being able to understand and characterize
the basic concepts of concurrency. In Petri nets, concurrency is a
fundamental concept in that Petri nets are inherently based on the idea
that behavior is (implicitly) concurrent unless explicitly
synchronized. This is in contrast to many other modeling formalisms
where concurrency must be explicitly introduced using parallel
composition operators. A further advantage of Petri nets is that they
rely on very few basic concepts, and are still able to model a wide
range of communication and synchronization concepts and patterns.\\

\vspace*{-1.5em}
%\noindent\rule{\columnwidth}{0.2em}
\paragraph*{\textsc{\textbf{SIDEBAR I: Petri nets}}}
A Petri net in its basic form is called a Place/Transition net (PTN) and
is a directed bi-partite graph with nodes consisting of places (drawn
as ellipses) and transitions (drawn as rectangles). The state of a
Petri net is called a marking and consists of a distribution of tokens
(drawn as black dots) positioned on the places. The execution of a
Petri net (also referred to as the token game) consists of occurrences
of enabled transitions removing tokens from input places and adding
tokens to output places as described by integer arc weights thereby
changing the current state (marking) of the Petri net. An abundance of
structural analysis methods (e.g., invariants and net reductions) as
well as dynamic analysis methods (e.g., state spaces and coverability
graphs) exists for Petri nets \cite{girault}\hfill$\qed$ \\
%\noindent\rule{\columnwidth}{0.2em}
% disadvantages - need for further development

In the decade following the introduction by C.~A. Petri, Petri nets
were widely accepted as one of the most well-founded theories to
describe important behavioral concepts such as concurrency, conflict
(non-determinism), synchronization, and resource sharing. Petri nets
were also used to model and analyze smaller concurrent
systems. However, the practical use soon revealed a serious
shortcoming. Petri nets (in their basic form) do not scale to large
systems unless one models the systems at a very high level of
abstraction. The primary reasons for this is that Petri nets are not
well-suited for modeling systems in which data and manipulation of
data plays a crucial role. Furthermore, Petri nets did not provide
concepts that made it easy to scale models according to some system
parameter, e.g., increase the number of servers in a modeled system
without having to make major changes to the model. This implied that
the use of Petri nets for practical modelling was decelerating. To
remedy this situation many researchers proposed different ad-hoc
extensions to Petri nets. This created a large variety of different
Petri net modeling languages. Many ad-hoc extensions were not
well-defined, and even when they were, they often had fundamental
theoretical problems. Whenever a new ad-hoc extension was introduced,
all the basic concepts and analysis methods had to be redefined -- to
apply for the extended Petri net language (with the ad-hoc
extension). With the invention of the first (text-based) computer
tools to support the analysis of Petri net models, the situation
became acute. Whenever a new ad-hoc extension was introduced (to
handle a modeling shortcoming) all existing computer tools became
void, and could only be used after time-consuming and error-prone
reprogramming. Hence, there was an urgent need to develop a class of
Petri nets that were general enough to handle a large variety of
different application areas without the need of making ad-hoc
extensions.

The first successful step towards a common more powerful class of
Petri nets were taken by Genrich and Lautenbach in 1979 with the
introduction of Predicate/Transition Nets (PrT nets)
\cite{genrich:81}. Their work was inspired by earlier work on
transition nets with \concept{colored tokens} by Schiffers and Wedde
\cite{schiffers:78} and transition nets with \concept{complex
  conditions} by Shapiro \cite{shapiro:78}. The basic idea behind PrT
nets was to introduce a set of colored tokens which can be
distinguished from each other -- in contrast to the indistinguishable
black tokens in basic Petri nets. In this way it became possible to
model different processes in a single subnet. PrT nets used arc
expressions to define how transitions can occur in different ways
(occurrence modes) depending on the colors of the involved input and
output tokens. The invention of colored distinguishable tokens in PrT
nets was a significant step forward -- but it still had some
limitations. PrT nets only had one set of token colors, and all places
had to use this set (or Cartesian products based on this set).

The second step towards a more general class of Petri nets was taken
by Jensen in his PhD thesis in 1980 with the introduction of the first
kind of Colored Petri Nets (CPNs) \cite{jensen:81}. This Petri net
model allowed the modeler to use a number of different color
sets. This made it possible to represent data values in a more
intuitive way instead of having to encode all data into a single
shared set. It later turned out to be convenient to define the color
sets by means of data types known from programming languages, such as
products, records, lists, and enumerations. The use of types had three
implications: Token colors became structured (and hence much more
powerful); type checking became possible (making it much easier to
locate modeling errors); and color sets, arc expressions and guards
could be specified by the well-known and powerful syntax and semantics
known from programming languages. This gave the modeler a convenient
way to handle complex data and specify the often complex interaction
between data and system behavior. A third step forward was taken by
Huber, Jensen, and Shapiro in 1990 with the introduction of
Hierarchical CPNs \cite{huber:91}. Their work was heavily inspired by
the hierarchy concepts in the Structured Analysis and Design Technique
(SADT) developed by Marca and McGowan \cite{sadt}. It was Shapiro who
got the idea to port the SADT hierarchy concepts to CPNs.  The
introduction of hierarchical CPNs allowed the modeler to structure a
large CPN model into a number of interacting and re-usable modules --
in a similar way as known from programming languages. This implied
that Petri net models of large systems become much more tractable,
since they can be split into modules of a reasonable size and the
model can be viewed at different levels of abstraction.


\ignore{
The shortcoming of ordinary Petri nets outlined above prompted a
research direction into the development of high-level Petri nets which
... KURT TO ADD HISTORICAL PERSPECTIVE ON THE DEVELOPMENT OF
HIGH-LEVEL NETS. WE NEED TO TALK ABOUT STANDARD ML SOMEWHERE AND SAY
THAT CPN ML IS BASED ON SML.
}
