\section{Introduction}

The vast majority of IT systems today can be characterized as
concurrent and distributed systems in that their operation inherently
relies on communication, synchronization, and resource sharing between
concurrently executing software components and applications. This is a
development that has been accelerated first with the pervasive
presence of the Internet as a communication infrastructure, and in
recent years by, e.g, cloud- and web-based services, mobile
applications, and multi-core computing architectures.

% main motivation and application domain of CPNs

The development of Coloured Petri Nets (CPNs) was initiated in the
early 80'es when distributed system was considered to become a major
paradigm for future computing systems. The goal of the CPN modeling
language was to develop a formally founded modeling language for
concurrent system that would make it possible to formally analyze and
validate concurrent systems, and which from a modeling perspective
would scale to industrial systems. A main motivation behind the
research into CPNs (and many other formal modeling languages) was
that the engineering of correct concurrent systems is a challenging
task due to their complex behavior which may result in subtle bugs. As
concurrent systems are becoming still more pervasive and critical to
society, formal techniques for concurrent system was -- and still is
-- a highly relevant technology to support the engineering of reliable
concurrent systems.

% historical perspectic on development and roots

At its very base, CPNs builds on the visionary work of C.~A Petri
\cite{X} who already in the 60'es introduced Petri Nets as a formalism
for concurrency and synchronization. In Petri Nets, concurrency is a
fundamental concept in that Petri Nets is inherently based on the idea
that everything is (implicitly) concurrent unless explicitly
synchronized. This is in contrast to many other modeling formalisms
where concurrency must be explicitly introduced using parallel
composition operators. A further advantage of Petri nets is that they
rely on very few basic concepts, and is still able to model a wide
range of communication and synchronization concepts and patterns. A
disadvantage of Petri nets in their basic form is, however, that they
do not scale to large systems unless one models the systems at a very
high level of abstraction. The primary reasons for this is that Petri
nets are not suited for modeling sequential computation and data
manipulation and they do not provide concepts that make it easy to
scale models according to some system parameter, e.g,. increase the
number of servers in a modeled system without having to make major
changes to the model.

\ignore{
The shortcoming of ordinary Petri nets outlined above prompted a
research direction into the development of high-level Petri nets which
... KURT TO ADD HISTORICAL PERSPECTIVE ON THE DEVELOPMENT OF
HIGH-LEVEL NETS. WE NEED TO TALK ABOUT STANDARD ML SOMEWHERE AND SAY
THAT CPN ML IS BASED ON SML.
}
