\paragraph*{\textsc{\textbf{SIDEBAR II: State Spaces and Model Checking}}}

State space exploration is the main technique used for verifying
behavioral properties of CPN models. In its basic form, a state space
is a directed graph consisting of a node for each reachable marking of
the CPN model and edges corresponding to occurrences of enabled
binding elements. From a constructed state space it is possible to
automatically verify a wide range of standard behavioral properties of
Petri nets (such as boundedness-, home-, and liveness properties). In
addition, state spaces of CPN models can be used to perform model
checking of behavioral properties expressed using temporal logics. The
main limitation of state space methods is the inherent state
exploration problem which implies that the full state space is often
too large to be explored with the available computing power. Many
advanced techniques exists to combat the state explosion problem, and
most of these can be applied also in the context of CPN models. \hfill
$\qed$

\paragraph*{\textsc{\textbf{SIDEBAR III: Performance Analysis}}}

Simulation is the main technique available for conducting performance
analysis of (timed) CPN models. The basic idea of simulation-based
performance analysis is to conduct a number of lengthy simulations
during which data on, e.g., queue length and delays, are
collected. Data collection is based on the concept of monitors that
observes simulations and write the extracted data into log files for
post processing or directly compute key performance figures such as
averages, standard deviation, and confidence intervals. In addition,
batch simulations can used to explore the parameter space of the model
and conduct multiple simulation for each parameter in order to obtain
statistically reliable results.\hfill $\qed$
