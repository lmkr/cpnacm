\paragraph*{\textsc{\textbf{Sidebar: State Spaces and Model Checking}}}

State space exploration is the main technique used for verifying
behavioral properties of CPN models. In its basic form, a state space
is a directed graph consisting of a node for each reachable marking
(state) of the CPN model and edges corresponding to occurrences of
enabled binding elements. From a constructed state space it is
possible to automatically verify a wide range of standard behavioral
properties of Petri nets (such as boundedness-, home-, and liveness
properties). In addition, state spaces of CPN models can be used to
perform model checking of behavioral properties expressed using
temporal logics such as CTL and LTL. The main limitation of state
space methods is the inherint state exploration problem which implies
that state spaces in their basic form are often too large to be with
the available computing power. Many advanced techqniues exists to
combat the state explosion problem, and most of these can be applied
also in the context of CPN models. \hfill $\qed$

\paragraph*{\textsc{\textbf{Sidebar: Performance Analysis}}}

Simulation-based performance analysis is the main technique available
for conducting quantative analysis of timed CPN models. The basic idea
of simulation-based performance analysis is to conduct a number of
lenghty simulations of the CPN model under consideration. During the
lengthy simulations data on, e.g., on queue length and delays, are
collected. Data collection is based on the concept of monitors that
observes simulations and write the extracted date into log files for
post processing or directly compute key performance figures such as
averages, standard deviation, and confidence intervals. In addition,
batch simulations can used to explore the parameter space of the model
and conduct multiple simulation for each parameter in order to obtain
statistically reliable results.\hfill $\qed$
