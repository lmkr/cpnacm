\vspace*{-0.75em}
\paragraph*{\textsc{\textbf{SIDEBAR II: State Spaces and Model Checking}}}

State space exploration is the main technique used for verifying
behavioral properties of CPN models. In its basic form, a state space
is a directed graph consisting of a node for each reachable marking of
the CPN model and edges corresponding to occurrences of enabled
binding elements. When conducting state space exploration, CPN Tools
explores all the enabled transition bindings in each encountered
marking. The expressive power of CPNs means that state spaces of CPN
models may be infinite. However, many CPN models arising in practice
do have a finite state space or can be modified (e.g., by bounding
data types and the number of tokens on places) to have a finite state
space. In this case, it is possible to automatically verify a wide
range of standard behavioral properties of CPN models (such as
boundedness, home, and liveness properties). In addition, state spaces
of CPN models can be used to perform model checking of behavioral
properties specified in temporal logic. The main limitation of state
space methods is the inherent state exploration problem which implies
that the full state space is often too large to be explored with the
available computing power. Many advanced techniques exist to combat
the state explosion problem, and most of these can be applied also in
the context of CPN models. The state space methods in CPN Tools have
been developed by Christensen, Kristensen, Westergaard, Evangelista,
and Mailund
\cite{sweep,asap}. 
 \hfill
$\qed$

\vspace*{-0.75em}
\paragraph*{\textsc{\textbf{SIDEBAR III: Timed Models and Performance Analysis}}}

CPNs also include a notion of time inspired by the work of van der
Aalst \cite{aalst:93}, which makes it possible to model the time taken
by different activities. This is based on the introduction of a
\concept{global clock} that represents the current model time and on
attaching \concept{time stamps} to tokens in addition to the token
colors. The time stamp of a token specifies the earliest model time at
which the token can be removed by the occurrence of a transition, and
delay inscriptions on the transitions and arcs are used to determine
the timestamps on tokens produced by transitions. During model
execution, the global model clock is always advanced to the earliest
next time at which a transition becomes enabled, and the model stays
at the current model time until no more transitions are enabled. The
time concept provides the foundation for conducting simulation-based
performance analysis of CPN models. The basic idea of simulation-based
performance analysis is to conduct a number of lengthy simulations
during which data on, e.g., queue length and delays, are
collected. Data collection is based on the concept of monitors that
observe simulations and write the extracted data into log files for
post processing or directly compute key performance figures such as
averages, standard deviations, and confidence intervals. In addition,
batch simulations can be used to explore the parameter space of the
model and conduct multiple simulations for each parameter in order to
obtain statistically reliable results. The support for performance
analysis is based on the work of Wells and Lindstr\o{}m
\cite{performance1}.\hfill $\qed$
