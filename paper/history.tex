\section{History of CPNs}

The basic ideas of Petri nets were introduced by Carl Adam Petri in
his doctoral thesis published in 1962 \cite{capetri:thesis}. This was
far ahead of the time where distributed systems were invented and
computers started to have parallel processes. At that time programs
and processing were considered to be sequential and
deterministic. Hence, it was extremely visionary of Carl Adam Petri to
predict the importance of being able to understand and characterize
the basic concepts of concurrency.

Over the next decade, Petri nets were widely accepted as one of the
most well-founded theories to describe important behavioral concepts
such as concurrency, conflict, synchronization and resource
sharing. Petri nets were also used to model and analyze small
practical systems. However, the practical use soon revealed a serious
shortcoming. Petri nets (in their basic form) are not well-suited to
model systems in which data play a crucial part. Since this is the
case for most computing systems (and many other kinds of systems) the
use of Petri nets for practical modeling were staggering. To remedy
this situation many modelers proposed different ad-hoc extensions to
Petri nets. This created a large "zoo" of different Petri net modeling
languages. Many ad-hoc extensions were not well-defined, and even when
they were, they introduced a fundamental problem. Whenever a new
ad-hoc extension was introduced, all the basic concepts and analysis
methods had to be redefined - to apply for the extended Petri net
language (with the ad-hoc extension).

With the invention of the first (text-based) computer tools to support
the analysis of Petri net models, the situation became acute. Whenever
a new ad-hoc extension was introduced (to handle a modeling problem)
all existing computer tools became void, and could only be used after
time-consuming (and error-prone) reprogramming/extension. Hence, there
was an urgent need to develop a class of Petri nets that were general
enough to handle a large variety of different application areas
without the need of making ad-hoc extensions.

The first successful step towards a common more powerful class of
Petri nets were taken by Hartmann Genrich and Kurt Lautenbach in 1979
with the introduction of Predicate/Transition Nets (PrT nets)
\cite{genrich:81}. Their work was inspired by earlier work on
transition nets with "colored tokens" by M. Schiffers and H. Wedde and
transition nets with "complex conditions" by Robert Shapiro. The basic
idea behind PrT nets was to introduce a set of colored tokens which
can be distinguished from each other - in contrast to the
indistinguishable black tokens in basic Petri nets. In this way it
became possible to model different processes in a single subnet. As an
example, consider the workers in the CPN model of
Sect.~\ref{sect:language}. In the basic Petri net model, it would be
necessary to have a subnet for each worker (even though they behave in
exactly the same way), but in PrT nets they can be modeled by a common
subnet, because we can use tokens with color \smlcode{wrk(1)} to model
the state of the first worker, tokens with the color \smlcode{wrk(2)}
to model the state of the second worker, and so on. This means that we
for \smlcode{W} workers can have a single \figitem{Idle} place, which
may contain tokens of \smlcode{W} different colors - instead of having
a separate \smlcode{Idle} place for each of the \smlcode{W}
workers. PrT nets use arc expressions to define how transitions can
occur in different ways (occurrence modes) depending of the colors of
the involved input and output tokens. This works in a similar way as
described for CPNs in Sect.~\ref{sect:language}.

The invention of colored distinguishable tokens in PrT nets was a
gigantic step forward - but it still had some limitations. PrT nets
only had one set of token colors, and all places have to use this set
(or Cartesian products of this set with itself). For our example in
Sect.~\ref{sect:language} this means that the identity workers, Yes/no
votes and abort/commit decisions all have to be modeled by colors in
this single unstructured set of token colors.

The second step towards a more general class of Petri nets was taken
by Kurt Jensen in his PhD thesis 1980 with the introduction of the
first kind of Colored Petri Nets \cite{jensen:81}]. This net model
  allowed the modeler to use a number of different color sets
  (e.g. one color set for the coordinator, a second for the workers, a
  third for Yes/no votes and a fourth for abort/commit
  decisions). This made it possible to represent data values in a more
  intuitive way instead of having to encode all data into a single
  shared set. It later turned out to be convenient to define the color
  sets by means of data types known from programming languages, such
  as products, records, lists, and enumerations. The use of types had
  three implications: Token colors became structured (and hence much
  more powerful). Type checking became possible (making it much easier
  to locate modeling errors). Color sets, arc expressions and guards
  could be specified by the well-known and powerful syntax and
  semantics known from programming languages. This gave the modeler a
  convenient way to handle complex data and specify the often complex
  interaction between data and system behavior.

A third step forward was taken by Peter Hubert, Kurt Jensen and Robert
Shapiro in 1990 with the introduction of Hierarchical CPNs
\cite{huber:91}. Their work was heavily inspired by the hierarchy
concepts in the Structured Analysis and DesignTechnique (SADT)
developed by D.A. Marca and C.L. McGowan. It was Robert Shapiro who
got the idea to port the SADT hierarchy concepts to CPN. Hierarchical
CPNs allows the modeler to split a large model into a number of
interacting and re-usable modules (components) - in a similar way as
used in many programming languages. The basic ideas are illustrated by
the modules, substitution transitions and port/socket places in
Sect.~\ref{sect:language}. The introduction of hierarchies in CPNs has
several implications: Petri net models of large systems become much
more tractable, since they can be split into modules of a reasonable
size (instead of working with a single monolithic net which has to be
glued to a large wall (or laid out on a football field). The human
modeler can concentrate on a few details (in a single module) at a
time - instead of being overwhelmed with the full details of the
complete model. CPN modules can be seen as black boxes, where
modelers, when they desire, can forget about the details within the
modules. This makes it possible to work at different abstraction
levels - and hence we talk about hierarchical CPNs. Finally there are
often system components that are used repeatedly. It would be
inefficient to model these components several times. Instead a module
can be defined once and used repeatedly. In this way there is only one
description to read, and one description to modify, when changes are
necessary.

The fourth step was taken by the creation of graphical computer tools
to support the modeling and analysis by means of CPNs. The Design/CPN
tool was created at Meta Software, Cambridge, Massachusetts, USA
starting in 1988 \cite{jensen:cpnmanual}. The main architects behind
the tool were Kurt Jensen, Robert Shapiro and Peter Hubert and the
implementation was made together with an international group of people
including Jawahar Malhotra (who got the brilliant idea to use the
Standard ML language for type definitions and net inscriptions), Ole
Bach Andersen (who implemented the graphical interface), Søren
Christensen (who implemented the complex algorithms for automatic
binding of free variables during simulations) and Hartmann Genrich
(who contributed with knowledge and experience from PrT nets). The
first version of Design/CPN supported modeling, syntax cheek and
interactive simulation. Later versions added timed CPNs (based on
ideas by Wil van der Aalst), fast automatic simulation (e.g. for
performance analysis) and state space analysis (to investigate
behavioral properties).

Steps 2-4 above are closely related. It was during the creation of the
Design/CPN tool that the need of modules was discovered and it was
also in this phase that it became clear that the type concept from
programming languages was extremely adequate for type definitions and
net inscriptions (instead of using more ad-hoc notations). If we
compare the development of CPNs with the development of programming
languages, we find many similarities. The first step from black tokens
to colored tokens corresponds to the step from bits to simple data
types. The second step corresponds to the invention of structured data
types and type checking. The third step corresponds to the
introduction of structuring concepts like modules, procedures,
functions and subroutines. The fourth step corresponds to the
implementation of compilers and run-time systems. Without these
programming languages are of no practical use. Analogously, a Petri
net language without tool support is useless for practical modeling
work.

The extensions of Petri nets and programing languages described above
add modeling power. They make it easier for the user to model/program
complex systems. The extensions do not add computational
power. Anything which can be programming in Java can (in theory) also
be programmed in assembler code. It is just much more time-consuming
and much more error-prone. Analogously it can be proved (quite easily)
that each hierarchical CPN can be unfolded to a (much larger) basic
Petri net (Place Transition Net) with exactly the same dynamic
behavior. Furthermore, each Place Transition Net can be folded into a
CPN with a single module consisting of a single place and a single
transition. In practice such a folding is totally uninteresting,
because the arc expressions will be extremely complex and totally
non-interpretable for a human being. However, the fact that the
unfolding and folding exists shows that hierarchical CPNs has the same
theoretical properties as basic Petri nets - in particular that CPNs
are a solid model for concurrency, conflict synchronization and
resource sharing.

The introduction of hierarchical CPNs supported by the Design/CPN tool
made a dramatic change to the practical used of Petri nets. The new
modeling language and its tool support were general and powerful
enough to eliminate the need of making ad-hoc extensions. A common
platform for practical modeling had been established and this was used
by most Petri net practitioners. The use of the platform was supported
by a three volume monograph on Colored Petri Nets published by Kurt
Jensen in 1992-1997 \cite{jensen:cpnvols}. In addition a large number
of research papers were published. Some of these describe new analysis
methods, while others describe experiences from practical modeling and
analysis. References to more than hundred modeling/analysis paper can
be found in \cite{cpnuse}. Many of the projects have been carried out
in an industrial environment.
 
Starting from year 2000 a second generation of tool support, called
CPN Tools, was designed and implemented at Aarhus University,
Denmark. The main architects behind the new tool were Kurt Jensen,
Søren Christensen and Michael Westergaard. The graphical user
interface was designed together with Michel Beaudouin-Lafon and Wendy
McKay from the international HCI research community. It is based on
empirical studies of the use of Design/CPN and much easier and
efficient to use. There is also a much faster simulation engine
(developed by Torben Haagh and Tommy Hansen). Many models run one
thousand times faster allowing complex automatic simulations to be
executed within seconds instead of hours. Many different kinds of
state space analysis are supported (developed by Lars Kristensen,
Michael Westergaard and Thomas Mailund). There is high-level support
for defining and collecting data from simulation-based performance
analysis (developed by Lisa Wells and Bo Lindstrøm). Finally there is
improved support for creating graphical feedback from ongoing
simulations (developed by Michael Westergaard).

An international standard for high-level Petri nets was developed and
approved in 2004 \cite{hcpnstandard}. The standardization work was
headed by Jonathan billington and the standard is heavily based on
CPNs (which adhere to the standard).

A new monograph on Colored Petri Nets has been published by Kurt
Jensen and Lars M. Kristensen in 2009 \cite{newcpnbook}. It provides
an in-depth, but yet compact introduction to modeling and validation
of concurrent systems by means of CPNs. The book introduces the
constructs of the CPN modeling language, presents its analysis
methods, and provides a comprehensive road map to the practical use of
CPNs. Furthermore, the book presents some selected industrial case
studies illustrating the practical use of CPN modeling and validation
for design, specification, simulation, and verification in a variety
of application domains. The book is aimed at use both in university
courses and for self-study.

In 2010 CPN Tools had 10.000 licenses in 150 countries. At that time
the development and maintenance of the tool set were transferred to
the group of Wil van der Aalst at the Technical University of
Eindhoven, The Netherlands \cite{cpntoolsweb}. New updates with
improved functionality are made at a regular basis.
